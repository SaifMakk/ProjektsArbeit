\section*{Kurzfassung}

Diese Arbeit entwickelt und evaluiert ein FreeRTOS-basiertes System zur akustischen
Entfernungsmessung zwischen zwei ESP32-S3-Knoten für den \emph{Außenbereich}. Ziel ist
eine zuverlässige und kostengünstige Kurzstreckenmessung trotz Wind, Umgebungsgeräuschen
und Mehrwegeffekten. Zwei identische Knoten mit I\textsuperscript{2}S-MEMS-Mikrofon und
 Lautsprecher tauschen hörbare Chirps im Ping-Pong-Verfahren aus; ein Funklink synchronisiert
 Startzeitpunkte und Zeitstempel. Zur Genauigkeitssteigerung kommen eine geeignete
 Chirp-Charakteristik im oberen Hörbereich, eine temperaturgestützte
 Schallgeschwindigkeitskorrektur, ein korrelationsbasiertes ToF-Verfahren mit
 Sub-Sample-Peak-Schätzung sowie eine einmalige Verzögerungskalibrierung zum Einsatz.
 Jitter wird durch ISR-Zeitstempel, I\textsuperscript{2}S-DMA und kerngebundene,
  priorisierte FreeRTOS-Tasks reduziert. Feldtests im Freien bestätigen die Eignung
  hörfrequenzbasierter Audio-ToF-Messungen und bilden die Grundlage für robustere
  Mehrknotensysteme.

\section*{Abstract}

This work develops and evaluates a FreeRTOS-based system for acoustic ranging between
two ESP32-S3 nodes for \emph{outdoor} use. The goal is reliable, low-cost short-range
measurements despite wind, ambient noise, and multipath. Two identical nodes equipped
with an I\textsuperscript{2}S MEMS microphone and a loudspeaker exchange audible chirps
in a ping-pong scheme; a radio link synchronizes start times and timestamps. To improve accuracy,
 the design employs a suitable chirp in the upper audible band, temperature-based speed-of-sound
 correction (on-board sensor), cross-correlation with sub-sample peak estimation, and a one-time
 delay calibration. Jitter is reduced through ISR-level timestamping, I\textsuperscript{2}S DMA,
  and core-pinned, prioritized FreeRTOS tasks. Outdoor field tests confirm the suitability of
  audible-band audio ToF and provide a basis for more robust multi-node systems.
