\section{Anforderungen}

Dieses Kapitel leitet aus den Anwendungsszenarien die funktionalen und nicht-funktionalen Anforderungen für das akustische \ac{ToF}-System ab.

\subsection{Anwendungsszenarien}

Das primäre Anwendungsszenario des entwickelten Systems ist die akustische Distanzmessung zwischen zwei im Außenraum positionierten Knoten. Jeder Knoten besteht aus einem ESP32-S3, einem Lautsprecher zur Signalerzeugung sowie einem Mikrofon zur Signalaufnahme. Die Kommunikation und Synchronisation zwischen den Knoten erfolgt drahtlos über Funk . 

\textbf{Funktionsweise:}  
Die Messung basiert auf einem zweistufigen Ablauf. Zunächst sendet der Sender-Knoten ein kurzes Funksignal an den Empfänger-Knoten. Dieses Signal dient der \textit{Synchronisation} und informiert den Empfänger über den unmittelbar bevorstehenden akustischen Sendevorgang. Direkt nach dem Aussenden des Funksignals erzeugt der Sender das akustische Chirp-Signal und gibt es über den Lautsprecher aus. Der Empfänger-Knoten startet exakt mit dem Eintreffen des Funksignals seine Aufnahme über das Mikrofon und zeichnet damit das ankommende Chirp-Signal auf. 

Die Zeitdifferenz zwischen dem Sendezeitpunkt (definiert durch das Funksignal) und dem Empfangszeitpunkt des akustischen Signals wird anschließend bestimmt. Dies geschieht durch Kreuzkorrelation des empfangenen Signals mit der bekannten Referenz des Chirps. Aus der Lage des Korrelationsmaximums ergibt sich die Laufzeitdifferenz, die unter Berücksichtigung der temperaturabhängigen Schallgeschwindigkeit in eine Distanz umgerechnet wird.  

Durch diese Kombination aus Funksynchronisation und akustischer Signalauswertung wird sichergestellt, dass die Messung nicht durch Taktabweichungen oder Jitter der beiden Knoten verfälscht wird. Das Funksignal reduziert die Unsicherheit beim Startzeitpunkt der Aufnahme, während das akustische Signal die eigentliche Distanzinformation liefert.  

\begin{figure}[h]
    \centering
    \includegraphics[width=0.9\linewidth]{Szenario mit zwei Knoten im Außenraum.png}
    \caption{Anwendungsszenario: Distanzmessung zwischen zwei Knoten mit Funksynchronisation und akustischem \ac{ToF}.}
    \label{fig:scenario}
\end{figure}


Das beschriebene Anwendungsszenario bildet die Grundlage für die nachfolgenden funktionalen und nicht-funktionalen Anforderungen.



\subsection{Funktionale Anforderungen}

\begin{verbatim}
-------------------------------------
1. Signalerzeugung (Chirp)
-------------------------------------
Description         : Der Sender muss akustische Chirp-Signale im Bereich 
                      von 2–8 \ac{kHz} mit einer Dauer von 20–60 ms 
                      zuverlässig erzeugen können.
Verification Method : Funktionstest der Signalerzeugung durch Messung mit
                      Oszilloskop und Spektralanalyse.
Category            : Functional

-------------------------------------
2. Signalaufnahme
-------------------------------------
Description         : Der Empfänger muss akustische Signale im gleichen Frequenzbereich 
                      mit Abtastraten bis 48 \ac{kHz} aufnehmen können.
Verification Method : Test der Mikrofon-Schnittstelle und Validierung der Abtastrate 
                      über Software- und Hardwaremessung.
Category            : Functional

-------------------------------------
3. Funksynchronisation
-------------------------------------
Description         : Das System muss eine drahtlose Synchronisation der Knoten 
                      (z. B. ESP-NOW) unterstützen, sodass der Empfänger zeitgenau 
                      mit der Aufnahme beginnt.
Verification Method : Messung der Zeitdifferenz zwischen Funksignal und Start 
                      der Aufnahme, Analyse mit Logic Analyzer.
Category            : Functional

-------------------------------------
4. Signalverarbeitung (Korrelation)
-------------------------------------
Description         : Der Empfänger muss die aufgenommenen Signale in Echtzeit 
                      mittels Kreuzkorrelation mit einer Referenz analysieren, 
                      um den Time-of-Flight zu bestimmen.
Verification Method : Softwaretest der Korrelation auf bekannten Referenzdaten, 
                      Validierung der Laufzeitbestimmung.
Category            : Functional

-------------------------------------
5. Distanzberechnung
-------------------------------------
Description         : Das System muss die Laufzeitdifferenz in eine Entfernung d 
                      umrechnen, unter Berücksichtigung der temperaturabhängigen 
                      Schallgeschwindigkeit.
Verification Method : Vergleich mit bekannten Referenzdistanzen unter verschiedenen 
                      Temperaturbedingungen.
Category            : Functional

-------------------------------------
6. Datenkommunikation
-------------------------------------
Description         : Die berechnete Distanz muss lokal gespeichert oder über Funk 
                      an andere Knoten oder eine zentrale Einheit übermittelt werden.
Verification Method : Funktionstest der Datenspeicherung und Funkübertragung 
                      zwischen den Knoten.
Category            : Functional

-------------------------------------
7. Messbereich
-------------------------------------
Description         : Das System muss Distanzen bis mindestens 25 m zuverlässig messen können.
Verification Method : Feldtests mit bekannten Referenzabständen im Bereich 1–25 m.
Category            : Functional
\end{verbatim}


\subsection{Nicht-funktionale Anforderungen}


\begin{verbatim}
-------------------------------------
1. Signalerzeugung (Chirp)
-------------------------------------
Description         : Der Sender muss akustische Chirp-Signale im Bereich 
                      von 2–8 \ac{kHz} mit einer Dauer von 20–60 ms 
                      zuverlässig erzeugen können.
Verification Method : Funktionstest der Signalerzeugung durch Messung mit 
                      Oszilloskop und Spektralanalyse.
Category            : Functional

-------------------------------------
2. Signalaufnahme
-------------------------------------
Description         : Der Empfänger muss akustische Signale im gleichen 
                      Frequenzbereich mit Abtastraten bis 48 \ac{kHz} 
                      aufnehmen können.
Verification Method : Test der Mikrofon-Schnittstelle und Validierung der 
                      Abtastrate über Software- und Hardwaremessung.
Category            : Functional

-------------------------------------
3. Funksynchronisation
-------------------------------------
Description         : Das System muss eine drahtlose Synchronisation der 
                      Knoten (z. B. ESP-NOW) unterstützen, sodass der 
                      Empfänger zeitgenau mit der Aufnahme beginnt.
Verification Method : Messung der Zeitdifferenz zwischen Funksignal und 
                      Start der Aufnahme, Analyse mit Logic Analyzer.
Category            : Functional

-------------------------------------
4. Signalverarbeitung (Korrelation)
-------------------------------------
Description         : Der Empfänger muss die aufgenommenen Signale in 
                      Echtzeit mittels Kreuzkorrelation mit einer Referenz 
                      analysieren, um den Time-of-Flight zu bestimmen.
Verification Method : Softwaretest der Korrelation auf bekannten 
                      Referenzdaten, Validierung der Laufzeitbestimmung.
Category            : Functional

-------------------------------------
5. Distanzberechnung
-------------------------------------
Description         : Das System muss die Laufzeitdifferenz in eine 
                      Entfernung d umrechnen, unter Berücksichtigung der 
                      temperaturabhängigen Schallgeschwindigkeit.
Verification Method : Vergleich mit bekannten Referenzdistanzen unter 
                      verschiedenen Temperaturbedingungen.
Category            : Functional

-------------------------------------
6. Datenkommunikation
-------------------------------------
Description         : Die berechnete Distanz muss lokal gespeichert oder 
                      über Funk an andere Knoten oder eine zentrale Einheit 
                      übermittelt werden.
Verification Method : Funktionstest der Datenspeicherung und 
                      Funkübertragung zwischen den Knoten.
Category            : Functional

-------------------------------------
7. Messbereich
-------------------------------------
Description         : Das System muss Distanzen bis mindestens 25 m 
                      zuverlässig messen können.
Verification Method : Feldtests mit bekannten Referenzabständen im 
                      Bereich 1–25 m.
Category            : Functional
\end{verbatim}


\subsection{Nicht-funktionale Anforderungen}

\begin{verbatim}
-------------------------------------
1. Robustheit gegenüber Störungen
-------------------------------------
Description         : Das System muss Umgebungsgeräusche (Wind, Verkehr, 
                      Tiere) und Mehrwegeffekte tolerieren und dennoch 
                      eine zuverlässige Peak-Erkennung ermöglichen.
Verification Method : Feldtests unter realen Outdoor-Bedingungen mit 
                      künstlich hinzugefügtem Störgeräusch.
Category            : Non-Functional

-------------------------------------
2. Messgenauigkeit
-------------------------------------
Description         : Die Distanzmessung soll eine Genauigkeit im Bereich 
                      von ± 0,1 m oder besser erreichen.
Verification Method : Vergleich mit Referenzmessungen mittels 
                      Laser-Distanzsensor.
Category            : Non-Functional

-------------------------------------
3. Energieeffizienz
-------------------------------------
Description         : Das System muss für batteriebetriebenen Einsatz 
                      optimiert sein, indem Rechen- und Funkoperationen 
                      energieeffizient geplant werden.
Verification Method : Messung des Stromverbrauchs während verschiedener 
                      Betriebsmodi.
Category            : Non-Functional

-------------------------------------
4. Echtzeitfähigkeit
-------------------------------------
Description         : Die gesamte Messung (Signalerzeugung, Aufnahme, 
                      Korrelation, Berechnung) soll innerhalb von weniger 
                      als 100 ms abgeschlossen sein.
Verification Method : Messung der End-to-End-Latenz mit Oszilloskop oder 
                      Logic Analyzer.
Category            : Non-Functional

-------------------------------------
5. Portabilität und Skalierbarkeit
-------------------------------------
Description         : Die Knoten sollen kompakt, leicht und in größeren 
                      Netzwerken einsetzbar sein.
Verification Method : Überprüfung des Formfaktors sowie Netzwerktests mit 
                      mehr als zwei Knoten.
Category            : Non-Functional

-------------------------------------
6. Wartungsarmut
-------------------------------------
Description         : Das System soll im Außenraum über längere Zeiträume 
                      ohne manuelle Justierung oder Kalibrierung zuverlässig 
                      arbeiten.
Verification Method : Langzeittest im Freien unter wechselnden 
                      Umweltbedingungen.
Category            : Non-Functional
\end{verbatim}


