\section{Anforderungen}

Dieses Kapitel leitet aus den Anwendungsszenarien die funktionalen und nicht-funktionalen Anforderungen für das akustische \ac{ToF}-System ab.

\subsection{Anwendungsszenarien}

Das primäre Anwendungsszenario des entwickelten Systems ist die akustische Distanzmessung zwischen zwei im Außenraum positionierten Knoten. Jeder Knoten besteht aus einem ESP32-S3, einem Lautsprecher zur Signalerzeugung sowie einem Mikrofon zur Signalaufnahme. Die Kommunikation und Synchronisation zwischen den Knoten erfolgt drahtlos über Funk . 

\textbf{Funktionsweise:}  
Die Messung basiert auf einem zweistufigen Ablauf. Zunächst sendet der Sender-Knoten ein kurzes Funksignal an den Empfänger-Knoten. Dieses Signal dient der \textit{Synchronisation} und informiert den Empfänger über den unmittelbar bevorstehenden akustischen Sendevorgang. Direkt nach dem Aussenden des Funksignals erzeugt der Sender das akustische Chirp-Signal und gibt es über den Lautsprecher aus. Der Empfänger-Knoten startet exakt mit dem Eintreffen des Funksignals seine Aufnahme über das Mikrofon und zeichnet damit das ankommende Chirp-Signal auf. 

Die Zeitdifferenz zwischen dem Sendezeitpunkt (definiert durch das Funksignal) und dem Empfangszeitpunkt des akustischen Signals wird anschließend bestimmt. Dies geschieht durch Kreuzkorrelation des empfangenen Signals mit der bekannten Referenz des Chirps. Aus der Lage des Korrelationsmaximums ergibt sich die Laufzeitdifferenz, die unter Berücksichtigung der temperaturabhängigen Schallgeschwindigkeit in eine Distanz umgerechnet wird.  

Durch diese Kombination aus Funksynchronisation und akustischer Signalauswertung wird sichergestellt, dass die Messung nicht durch Taktabweichungen oder Jitter der beiden Knoten verfälscht wird. Das Funksignal reduziert die Unsicherheit beim Startzeitpunkt der Aufnahme, während das akustische Signal die eigentliche Distanzinformation liefert.  

\begin{figure}[h]
    \centering
    \includegraphics[width=0.9\linewidth]{Szenario mit zwei Knoten im Außenraum.png}
    \caption{Anwendungsszenario: Distanzmessung zwischen zwei Knoten mit Funksynchronisation und akustischem \ac{ToF}.}
    \label{fig:scenario}
\end{figure}


Das beschriebene Anwendungsszenario bildet die Grundlage für die nachfolgenden funktionalen und nicht-funktionalen Anforderungen.



\subsection{Funktionale Anforderungen}

\begin{verbatim}
F1: Signalerzeugung (Chirp)
-----------------------------------------------
Description    : Akustische Chirp-Signale (2-8 kHz, 
                 20-60 ms) zuverlässig erzeugen
Verification   : Funktionstest mit Oszilloskop und 
                 Spektralanalyse
Category       : Functional

F2: Signalaufnahme
-----------------------------------------------
Description    : Akustische Signale mit Abtastraten 
                 bis 48 kHz aufnehmen
Verification   : Test der Mikrofon-Schnittstelle und
                 Abtastrate-Validierung
Category       : Functional

F3: Funksynchronisation
-----------------------------------------------
Description    : Drahtlose Synchronisation der Knoten
                 (ESP-NOW) für zeitgenaue Aufnahme
Verification   : Zeitdifferenz-Messung mit 
                 Logic Analyzer
Category       : Functional

F4: Signalverarbeitung (Korrelation)
-----------------------------------------------
Description    : Echtzeitanalyse mittels 
                 Kreuzkorrelation für ToF-Bestimmung
Verification   : Softwaretest mit Referenzdaten
Category       : Functional

F5: Distanzberechnung
-----------------------------------------------
Description    : Umrechnung Laufzeitdifferenz in 
                 Entfernung (temperaturabhängige 
                 Schallgeschwindigkeit)
Verification   : Vergleich mit Referenzdistanzen bei
                 verschiedenen Temperaturen
Category       : Functional

F6: Datenkommunikation
-----------------------------------------------
Description    : Lokale Speicherung oder 
                 Funkübertragung der Distanz
Verification   : Test der Datenspeicherung und 
                 Funkübertragung
Category       : Functional

F7: Messbereich
-----------------------------------------------
Description    : Distanzmessung bis mindestens 25 m
Verification   : Feldtests mit Referenzabständen 
                 (1-25 m)
Category       : Functional
\end{verbatim}


\subsection{Nicht-funktionale Anforderungen}

\begin{verbatim}
NF1: Robustheit gegenüber Störungen
-----------------------------------------------
Description    : Toleranz gegenüber Umgebungsgeräuschen
                 (Wind, Verkehr) und Mehrwegeffekten
Verification   : Feldtests mit künstlichem Störgeräusch
Category       : Non-Functional

NF2: Messgenauigkeit
-----------------------------------------------
Description    : Distanzmessung mit ± 0,1 m Genauigkeit
Verification   : Vergleich mit Laser-Distanzsensor
Category       : Non-Functional

NF3: Energieeffizienz
-----------------------------------------------
Description    : Optimierung für batteriebetriebenen 
                 Einsatz (energieeffiziente Operationen)
Verification   : Stromverbrauch-Messung in verschiedenen
                 Betriebsmodi
Category       : Non-Functional

NF4: Echtzeitfähigkeit
-----------------------------------------------
Description    : Gesamte Messung unter 100 ms 
                 (Erzeugung bis Berechnung)
Verification   : End-to-End-Latenz-Messung mit 
                 Oszilloskop oder Logic Analyzer
Category       : Non-Functional

NF5: Portabilität und Skalierbarkeit
-----------------------------------------------
Description    : Kompakte Knoten für größere Netzwerke
Verification   : Formfaktor-Überprüfung und 
                 Netzwerktests
Category       : Non-Functional

NF6: Wartungsarmut
-----------------------------------------------
Description    : Zuverlässiger Betrieb im Außenraum 
                 ohne manuelle Justierung
Verification   : Langzeittest unter wechselnden 
                 Umweltbedingungen
Category       : Non-Functional
\end{verbatim}


