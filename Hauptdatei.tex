% Festlegung des Allgemeinen Dokumentenformats
\documentclass[a4paper,12pt,headsepline]{scrartcl}

% Umlaute unter UTF8 nutzen
\usepackage[utf8]{inputenc}

% Variablen
% Variablen welche innerhalb der gesamten Arbeit zur Verfügung stehen sollen
% Titel und Thema
\newcommand{\titleDocument}{Projektarbeit}
\newcommand{\subjectDocument}{Entwicklung eines FreeRTOS-basierten Systems zur akustischen Entfernungsmessung im hörbaren Frequenzbereich}

% Dokumenttyp und Abschlussgrad
\newcommand{\documentType}{Projektarbeit}
\newcommand{\degree}{Bachelor of Science}
\newcommand{\degreeShort}{B. Sc.}

% Institution
\newcommand{\university}{An der Fachhochschule Dortmund}
\newcommand{\faculty}{im Fachbereich Informatik}
\newcommand{\program}{Studiengang Technische Informatik}

% Autorendaten
\newcommand{\authorName}{seifeddine Makhlouf}
\newcommand{\authorBirthdate}{20.01.2001}
\newcommand{\studentId}{999999}
\newcommand{\city}{Dortmund}

% Betreuung
\newcommand{\advisorA}{Prof. Dr. Frank Künemund}
\newcommand{\advisorB}{Dipl.-Ing. Dieter Zumkehr}


% weitere Pakete
% Grafiken aus PNG Dateien einbinden
\usepackage{graphicx}
\usepackage{tikz}

% Deutsche Sonderzeichen und Silbentrennung nutzen
\usepackage[ngerman]{babel}

% Eurozeichen einbinden
\usepackage[right]{eurosym}

% Zeichenencoding
\usepackage[T1]{fontenc}

\usepackage{lmodern}

% floatende Bilder ermöglichen
%\usepackage{floatflt}

\usepackage{float}

% mehrseitige Tabellen ermöglichen
\usepackage{longtable}

\usepackage{diagbox}
\usepackage{colortbl}

% Unterstützung für Schriftarten
%\newcommand{\changefont}[3]{ 
%\fontfamily{#1} \fontseries{#2} \fontshape{#3} \selectfont}

% Packet für Seitenrandabständex und Einstellung für Seitenränder
\usepackage{geometry}
\geometry{left=3.5cm, right=2cm, top=2.5cm, bottom=2cm}

% Paket für Boxen im Text
\usepackage{fancybox}

% bricht lange URLs "schön" um
\usepackage[hyphens,obeyspaces,spaces]{url}

% Paket für Textfarben
\usepackage{color}

% Mathematische Symbole importieren
\usepackage{amssymb}

\usepackage{amsmath}

% auf jeder Seite eine Überschrift (alt, zentriert)
%\pagestyle{headings}

\usepackage{pdfpages}
\usepackage{etoolbox}

% erzeugt Inhaltsverzeichnis mit Querverweisen zu den Abschnitten (PDF Version)
\usepackage[bookmarksnumbered,pdftitle={\titleDocument},hyperfootnotes=false]{hyperref}
%\hypersetup{colorlinks, citecolor=red, linkcolor=blue, urlcolor=black}
%\hypersetup{colorlinks, citecolor=black, linkcolor= black, urlcolor=black}

% neue Kopfzeilen mit fancypaket
\usepackage{fancyhdr} %Paket laden
\pagestyle{fancy} %eigener Seitenstil
\fancyhf{} %alle Kopf- und Fußzeilenfelder bereinigen
\fancyhead[L]{\nouppercase{\leftmark}} %Kopfzeile links
\fancyhead[C]{} %zentrierte Kopfzeile
\fancyhead[R]{\thepage} %Kopfzeile rechts
\renewcommand{\headrulewidth}{0.4pt} %obere Trennlinie
%\fancyfoot[C]{\thepage} %Seitennummer
%\renewcommand{\footrulewidth}{0.4pt} %untere Trennlinie

% change font to serif for headings - buggy
% \usepackage{sectsty}
% \allsectionsfont{\normalfont\bfseries}

% für Tabellen
\usepackage{array}

% Runde Klammern für Zitate
%\usepackage[numbers,round]{natbib}

% Festlegung Art der Zitierung - Havardmethode: Abkuerzung Autor + Jahr
\bibliographystyle{alphadin}

% Schaltet den zusätzlichen Zwischenraum ab, den LaTeX normalerweise nach einem Satzzeichen einfügt.
%\frenchspacing

% Paket für Zeilenabstand
\usepackage{setspace}

% für Bildbezeichner
\usepackage{capt-of}

% für Stichwortverzeichnis
\usepackage{makeidx}

%Konfiguriere das Inhaltsverzeichnis
\usepackage{tocbasic}
\DeclareTOCStyleEntries[
  raggedentrytext,
  numwidth=0pt,
  numsep=1ex,
  dynnumwidth,
]{tocline}{chapter,section,subsection,subsubsection,paragraph,subparagraph}
\DeclareTOCStyleEntries[
  indent=0pt,
  linefill=\TOCLineLeaderFill,
]{tocline}{section,subsection,subsubsection,paragraph,subparagraph}


% für Listings
% \usepackage{listings}
% \lstset{numbers=left, numberstyle=\tiny, numbersep=5pt, keywordstyle=\color{black}\bfseries, stringstyle=\ttfamily,showstringspaces=false,basicstyle=\footnotesize,captionpos=b}
\usepackage{xcolor}
\usepackage[newfloat]{minted}
\usepackage{caption}
\usepackage{fancyhdr}
\newenvironment{code}{\captionsetup{type=listing}}{}
\SetupFloatingEnvironment{listing}{name=Quellcode}
\setminted{
    linenos,
    frame=single,
    bgcolor=black!5,
    fontsize=\footnotesize
}
\usepackage{graphicx}  % Add this to your preamble

\usepackage[withpage]{acronym}

% Absatz ohne Einrückung
\setlength{\parskip}{1em} % 1em entspricht der Breite eines 'M'-Zeichens
\setlength{\parindent}{0pt}

% Indexerstellung
\makeindex

% Abkürzungsverzeichnis
\usepackage[german]{nomencl}
\let\abbrev\nomenclature

% Abkürzungsverzeichnis LiveTex Version
% Titel des Abkürzungsverzeichnisses
\renewcommand{\nomname}{Abkürzungsverzeichnis}
% Abstand zwischen Abkürzung und Erläuterung
\setlength{\nomlabelwidth}{.25\textwidth}
% Zwischenraum zwischen Abkürzung und Erläuterung mit Punkten
\renewcommand{\nomlabel}[1]{#1 \dotfill}
% Variation des Abstandes der einzelnen Abkürzungen zu einander
\setlength{\nomitemsep}{-\parsep}
% Index mit Abkürzungen erzeugen
\makenomenclature
%\makeglossary

% Abkürzungsverzeichnis TeTEX Version
% \usepackage[german]{nomencl}
% \makenomenclature
% %\makeglossary
% \renewcommand{\nomname}{Abkürzungsverzeichnis}
% \AtBeginDocument{\setlength{\nomlabelwidth}{.25\columnwidth}}
% \renewcommand{\nomlabel}[1]{#1 \dotfill}
% \setlength{\nomitemsep}{-\parsep}

% Optional: Einzelne Zeilen am Anfang einer Seite unterdrücken (Schusterjungen)
% \clubpenalty = 10000
% Optional: Einzelne Zeilen am Ende einer Seite unterdrücken (Hurenkinder)
% \widowpenalty = 10000
% \displaywidowpenalty = 10000

\begin{document}
% hier werden die Trennvorschläge inkludiert
\input{latex_einstellungen/trennung}

% Schriftart Helvetica verwenden
%\usepackage{helvet}
%\renewcommand\familydefault{\sfdefault}

% Leere Seite am Anfang
%\thispagestyle{empty} % erzeugt Seite ohne Kopf- / Fusszeile
%\mbox{}
%\newpage

% Titelseite %
% Cover page (Deckblatt)
% Uses variables defined in latex_einstellungen/variablen.tex
\thispagestyle{empty}

% Optional: Logo (place your file under assets/ and uncomment)
%\begin{figure}[t]
%  \centering
%  \includegraphics[width=0.25\textwidth]{assets/logo.pdf}
%\end{figure}

\vspace*{1.2cm}
\begin{center}
    {\normalsize \documentType}
\end{center}

\vspace{2.8cm}
\begin{center}
   
    {\Large \subjectDocument}\\[0.8em]
\end{center}

\vspace{5.5em}
\begin{center}
    \begingroup
    \small
    \ifdefempty{\university}{}{\university\par}
    \ifdefempty{\faculty}{}{\faculty\par}
    \ifdefempty{\program}{}{\program\par}
    
    \endgroup
\end{center}

\vspace{3em}
\begin{center}
    \textbf{Autor/in}\\
    \authorName\\
    geboren am \authorBirthdate\\
    Matrikelnummer: \studentId
\end{center}

\vspace{2em}
\begin{flushleft}
\begin{tabular}{@{}ll}
\textbf{Betreuung durch:} & \advisorA \\
                          & \advisorB \\
\end{tabular}
\end{flushleft}

\vfill

\begin{flushleft}
\begin{tabular}{@{}ll}
\textbf{Version vom:}    & \city, \today \\
\end{tabular}
\end{flushleft}


% römische Numerierung
\pagenumbering{Roman}

% 1.5 facher Zeilenabstand
\onehalfspacing

\newpage

% Einleitung / Abstract
\thispagestyle{empty}
\section*{Kurzfassung}

Kurze Zusammenfassung der Arbeit in deutscher Sprache. Beschreiben Sie in 5–8
Sätzen Motivation, Vorgehen, Ergebnisse und Schlussfolgerungen. Ersetzen Sie
diesen Platzhaltertext durch Ihre eigene Kurzfassung.

\section*{Abstract}

Short summary of the work in English. Describe motivation, approach, results,
and conclusions in 5–8 sentences. Replace this placeholder text with your own
abstract.


% einfacher Zeilenabstand
\singlespacing

\newpage
% Seitenzählung bei Inhaltsverzeichnis beginnen
\setcounter{page}{1}

% Inhaltsverzeichnis anzeigen
\thispagestyle{empty}
\vspace*{1.5cm}
\begin{center}
{\Large \textbf{Erweitertes Inhaltsverzeichnis}}
\end{center}

\vspace{2cm}
\noindent
\textbf{Kurzfassung} \dotfill II

\noindent
\textbf{Abstract} \dotfill III

\vspace{0.5cm}
\noindent
\textbf{Einleitung} \dotfill 1

\noindent
\hspace{1cm} 3.1 Motivation \dotfill 1

\noindent
\hspace{1cm} 3.2 Zielsetzung \dotfill 2

\noindent
\hspace{1cm} 3.3 Aufbau der Arbeit \dotfill 3

\vspace{0.5cm}
\noindent
\textbf{Grundlagen} \dotfill 4

\noindent
\hspace{1cm} 4.1 Schallausbreitung im Außenraum \dotfill 4

\noindent
\hspace{1cm} 4.2 Signalgestaltung für akustische ToF-Verfahren \dotfill 6

\noindent
\hspace{1cm} 4.3 Korrelationsmethoden \dotfill 8

\noindent
\hspace{1cm} 4.4 Eigenschaften von ESP32‑S3 und FreeRTOS \dotfill 10

\vspace{0.5cm}
\noindent
\textbf{Anforderungen} \dotfill 12

\noindent
\hspace{1cm} 5.1 Anwendungsszenarien \dotfill 12

\noindent
\hspace{1cm} 5.2 Funktionale Anforderungen  \dotfill 14

\noindent
\hspace{1cm} 5.3 Nicht-funktionale Anforderungen  \dotfill 16

\vspace{0.5cm}
\noindent
\textbf{Hardware} \dotfill 18

\noindent
\hspace{1cm} 6.1 Prototypenaufbau (Steckbrett) \dotfill 18

\noindent
\hspace{1cm} 6.2 Lautsprecher- und Mikrofonwahl \dotfill 20

\noindent
\hspace{1cm} 6.3 Funkmodulauswahl \dotfill 22


\noindent
\hspace{1cm} 6.6 Schaltplan (Schematik) \dotfill 28

\noindent
\hspace{1cm} 6.7 PCB-Layout \dotfill 30

\vspace{0.5cm}
\noindent
\textbf{Software} \dotfill 28

\noindent
\hspace{1cm} 7.1 FreeRTOS-Architektur  \dotfill 28

\noindent
\hspace{1cm} 7.2 I²S-DMA \dotfill 29

\noindent
\hspace{1cm} 7.3 Funkbasierte Synchronisation \dotfill 30

\noindent
\hspace{1cm} 7.4 Maßnahmen zur Jitterreduktion \dotfill 31

\vspace{0.5cm}
\noindent
\textbf{Signaldesign} \dotfill 32

\noindent
\hspace{1cm} 8.1 Chirp-Charakteristik, Fensterung und Bandbegrenzung \dotfill 32

\noindent
\hspace{1cm} 8.2 Vorverarbeitung \dotfill 33

\noindent
\hspace{1cm} 8.3 Korrelationsbasierte Laufzeitschätzung mit Sub-Sample-Peak-Refinement \dotfill 34

\noindent
\hspace{1cm} 8.4 Temperaturkompensation \dotfill 35

\noindent
\hspace{1cm} 8.5 Verzögerungskalibrierung \dotfill 36

\vspace{0.5cm}
\noindent
\textbf{Methodik} \dotfill 37



\vspace{0.5cm}
\noindent
\textbf{Ergebnisse} \dotfill 52

%

\vspace{0.5cm}
\noindent
\textbf{Zusammenfassung und Ausblick} \dotfill 64

\vspace{0.8cm}
\noindent
\textbf{Abbildungsverzeichnis} \dotfill II

\noindent
\textbf{Tabellenverzeichnis} \dotfill III

\noindent
\textbf{Quellcodeverzeichnis} \dotfill IV

\noindent
\textbf{Abkürzungsverzeichnis} \dotfill V

\noindent
\textbf{Literaturverzeichnis} \dotfill I

\noindent
\textbf{Anhang} \dotfill VI

\newpage
% das Abbildungsverzeichnis
% Verion 1: Abbildungsverzeichnis MIT führender Nummberierung endgueltig anzeigen
\listoffigures
% Abbildungsverzeichnis soll im Inhaltsverzeichnis auftauchen
\addcontentsline{toc}{section}{Abbildungsverzeichnis}

% Verion 2: Abbildungsverzeichnis OHNE führende Nummberierung endgueltig anzeigen
%\begingroup
%\renewcommand\numberline[1]{}
%\listoffigures
%\endgroup


% das Tabellenverzeichnis
\newpage
% \fancyhead[L]{Abbildungsverzeichnis / Abkürzungsverzeichnis} %Kopfzeile links
% Tabellenverzeichnis endgültig anzeigen
\listoftables
% Tabellenverzeichnis soll im Inhaltsverzeichnis auftauchen
\addcontentsline{toc}{section}{Tabellenverzeichnis}

% das Quellcodeverzeichnis
\newpage
\renewcommand*{\listlistingname}{Quellcodeverzeichnis}
\listoflistings % Add Quellcodeverzeichnis
\addcontentsline{toc}{section}{Quellcodeverzeichnis}

% das Abkürzungsverzeichnis
\newpage
% das Abkürzungsverzeichnis ausgeben
\fancyhead[L]{Abkürzungsverzeichnis} %Kopfzeile links
\section*{Abkürzungsverzeichnis}

\begin{acronym}
    \acro{API}{Application Programming Interface}
    \acro{CPU}{Central Processing Unit}
    \acro{ToF}{Time-of-Flight}
    \acro{SNR}{Signal-Rausch-Abstand}
    \acro{GCC}{Generalized Cross-Correlation}
    \acro{DSP}{Digital Signal Processing}
    \acro{I2S}{Inter-IC Sound}
    \acro{ADC}{Analog-to-Digital Converter}
    \acro{DAC}{Digital-to-Analog Converter}
    \acro{MEMS}{Micro-Electro-Mechanical Systems}
    \acro{FFT}{Fast Fourier Transform}
    \acro{WLAN}{Wireless Local Area Network}
    \acro{Bluetooth}{Bluetooth Low Energy}
    \acro{IEEE}{Institute of Electrical and Electronics Engineers}
    % Fügen Sie weitere Abkürzungen hier hinzu
\end{acronym}



% \printnomenclature[3cm]
% Abkürzungsverzeichnis soll im Inhaltsverzeichnis auftauchen
\addcontentsline{toc}{section}{Abkürzungsverzeichnis}


%%%%%%% EINLEITUNG %%%%%%%%%%%%
\newpage
\fancyhead[L]{\nouppercase{\leftmark}} %Kopfzeile links

% 1,5 facher Zeilenabstand
\onehalfspacing

% arabische Seitennummerierung ab hier
\pagenumbering{arabic}

% Alternative Einbindung des Abstract in Kapitel "0" falls gewünscht
%\setcounter{section}{-1}
%\setcounter{page}{0}

% Option: Einbindung abstract
%\section*{Kurzfassung}

Kurze Zusammenfassung der Arbeit in deutscher Sprache. Beschreiben Sie in 5–8
Sätzen Motivation, Vorgehen, Ergebnisse und Schlussfolgerungen. Ersetzen Sie
diesen Platzhaltertext durch Ihre eigene Kurzfassung.

\section*{Abstract}

Short summary of the work in English. Describe motivation, approach, results,
and conclusions in 5–8 sentences. Replace this placeholder text with your own
abstract.

%\newpage

% einzelne Kapitel werden hier eingebunden
\section{Einleitung}

Führen Sie in das Thema ein, motivieren Sie die Arbeit und formulieren Sie die
Ziele. Skizzieren Sie den Aufbau des Dokuments. Dieser Text dient als
Platzhalter und sollte vollständig durch Ihre eigene Einleitung ersetzt werden.

\newpage

\section{Grundlagen}

Dieses Kapitel stellt die theoretischen und technischen Grundlagen vor, die zum Verständnis der akustischen Time-of-Flight-Messung im Außenbereich erforderlich sind.

\subsection{Schallausbreitung im Außenraum}

\subsection{Signalgestaltung für akustische ToF-Verfahren}

\subsection{Korrelationsmethoden}


\subsection{Eigenschaften von ESP32-S3 und FreeRTOS}


\newpage

\section{Vorbereitung}

Beschreiben Sie hier die Vorarbeiten, Annahmen, Anforderungen und den
Projektkontext. Listen Sie relevante Werkzeuge, Datenquellen und Rahmenbedingungen
auf. Dieser Platzhaltertext sollte durch Ihre eigenen Inhalte ersetzt werden.

\newpage

\section{Umsetzung}

Beschreiben Sie die Implementierungsschritte, Architektur, Schnittstellen und
wesentliche Designentscheidungen. Referenzieren Sie Abbildungen, Tabellen und
Codeausschnitte nach Bedarf.



\newpage

\section{Evaluation}

Beschreiben Sie den Evaluationsaufbau, die Metriken und die Auswertung der
Ergebnisse. Fügen Sie ggf. Abbildungen und Tabellen hinzu.



\newpage

\section{Zusammenfassung}

Fassen Sie die Arbeit kurz zusammen, nennen Sie die wichtigsten Beiträge und
Ergebnisse und skizzieren Sie mögliche zukünftige Arbeiten. Ersetzen Sie diesen
Platzhalter durch Ihre eigene Zusammenfassung.

\newpage

\pagenumbering{Roman}
% einfacher Zeilenabstand
\singlespacing
% Literaturliste soll im Inhaltsverzeichnis auftauchen
\newpage
\phantomsection
\addcontentsline{toc}{section}{Literaturverzeichnis}
% Literaturverzeichnis anzeigen
\renewcommand\refname{Literaturverzeichnis}
\bibliography{Hauptdatei}

%% Index soll Stichwortverzeichnis heissen
% \newpage
% % Stichwortverzeichnis soll im Inhaltsverzeichnis auftauchen
% \addcontentsline{toc}{section}{Stichwortverzeichnis}
% \renewcommand{\indexname}{Stichwortverzeichnis}
% % Stichwortverzeichnis endgültig anzeigen
% \printindex

% \onehalfspacing
% % evtl. Anhang
% \newpage
% \phantomsection
% \addcontentsline{toc}{section}{Anhang}
% \fancyhead[L]{Anhang} %Kopfzeile links
% \input{anhang}

% \newpage
% \includepdf[pages=1, fitpaper=true]{flattened.pdf}

% \fancyhead[L]{Eigenständigkeitserklährung} %Kopfzeile links
% \input{eigenstaendigkeitserklaehrung} % deleted 

\end{document}
