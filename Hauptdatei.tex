% Festlegung des Allgemeinen Dokumentenformats
\documentclass[a4paper,12pt,headsepline]{scrartcl}

% Umlaute unter UTF8 nutzen
\usepackage[utf8]{inputenc}

% Variablen
% Variablen welche innerhalb der gesamten Arbeit zur Verfügung stehen sollen
% Titel und Thema
\newcommand{\titleDocument}{Projektarbeit}
\newcommand{\subjectDocument}{Entwicklung eines FreeRTOS-basierten Systems zur akustischen Entfernungsmessung im hörbaren Frequenzbereich}

% Dokumenttyp und Abschlussgrad
\newcommand{\documentType}{Projektarbeit}
\newcommand{\degree}{Bachelor of Science}
\newcommand{\degreeShort}{B. Sc.}

% Institution
\newcommand{\university}{An der Fachhochschule Dortmund}
\newcommand{\faculty}{im Fachbereich Informatik}
\newcommand{\program}{Studiengang Technische Informatik}

% Autorendaten
\newcommand{\authorName}{seifeddine Makhlouf}
\newcommand{\authorBirthdate}{20.01.2001}
\newcommand{\studentId}{999999}
\newcommand{\city}{Dortmund}

% Betreuung
\newcommand{\advisorA}{Prof. Dr. Frank Künemund}
\newcommand{\advisorB}{Dipl.-Ing. Dieter Zumkehr}


% weitere Pakete
% Grafiken aus PNG Dateien einbinden
\usepackage{graphicx}
\usepackage{tikz}

% Deutsche Sonderzeichen und Silbentrennung nutzen
\usepackage[ngerman]{babel}

% Eurozeichen einbinden
\usepackage[right]{eurosym}

% Zeichenencoding
\usepackage[T1]{fontenc}

\usepackage{lmodern}

% floatende Bilder ermöglichen
%\usepackage{floatflt}

\usepackage{float}

% mehrseitige Tabellen ermöglichen
\usepackage{longtable}

\usepackage{diagbox}
\usepackage{colortbl}

% Unterstützung für Schriftarten
%\newcommand{\changefont}[3]{ 
%\fontfamily{#1} \fontseries{#2} \fontshape{#3} \selectfont}

% Packet für Seitenrandabständex und Einstellung für Seitenränder
\usepackage{geometry}
\geometry{left=3.5cm, right=2cm, top=2.5cm, bottom=2cm}

% Paket für Boxen im Text
\usepackage{fancybox}

% bricht lange URLs "schön" um
\usepackage[hyphens,obeyspaces,spaces]{url}

% Paket für Textfarben
\usepackage{color}

% Mathematische Symbole importieren
\usepackage{amssymb}

\usepackage{amsmath}

% auf jeder Seite eine Überschrift (alt, zentriert)
%\pagestyle{headings}

\usepackage{pdfpages}
\usepackage{etoolbox}

% erzeugt Inhaltsverzeichnis mit Querverweisen zu den Abschnitten (PDF Version)
\usepackage[bookmarksnumbered,pdftitle={\titleDocument},hyperfootnotes=false]{hyperref}
%\hypersetup{colorlinks, citecolor=red, linkcolor=blue, urlcolor=black}
%\hypersetup{colorlinks, citecolor=black, linkcolor= black, urlcolor=black}

% neue Kopfzeilen mit fancypaket
\usepackage{fancyhdr} %Paket laden
\pagestyle{fancy} %eigener Seitenstil
\fancyhf{} %alle Kopf- und Fußzeilenfelder bereinigen
\fancyhead[L]{\nouppercase{\leftmark}} %Kopfzeile links
\fancyhead[C]{} %zentrierte Kopfzeile
\fancyhead[R]{\thepage} %Kopfzeile rechts
\renewcommand{\headrulewidth}{0.4pt} %obere Trennlinie
%\fancyfoot[C]{\thepage} %Seitennummer
%\renewcommand{\footrulewidth}{0.4pt} %untere Trennlinie

% change font to serif for headings - buggy
% \usepackage{sectsty}
% \allsectionsfont{\normalfont\bfseries}

% für Tabellen
\usepackage{array}

% Runde Klammern für Zitate
%\usepackage[numbers,round]{natbib}

% Festlegung Art der Zitierung - Havardmethode: Abkuerzung Autor + Jahr
\bibliographystyle{alphadin}

% Schaltet den zusätzlichen Zwischenraum ab, den LaTeX normalerweise nach einem Satzzeichen einfügt.
%\frenchspacing

% Paket für Zeilenabstand
\usepackage{setspace}

% für Bildbezeichner
\usepackage{capt-of}

% für Stichwortverzeichnis
\usepackage{makeidx}

%Konfiguriere das Inhaltsverzeichnis
\usepackage{tocbasic}
\DeclareTOCStyleEntries[
  raggedentrytext,
  numwidth=0pt,
  numsep=1ex,
  dynnumwidth,
]{tocline}{chapter,section,subsection,subsubsection,paragraph,subparagraph}
\DeclareTOCStyleEntries[
  indent=0pt,
  linefill=\TOCLineLeaderFill,
]{tocline}{section,subsection,subsubsection,paragraph,subparagraph}


% für Listings
% \usepackage{listings}
% \lstset{numbers=left, numberstyle=\tiny, numbersep=5pt, keywordstyle=\color{black}\bfseries, stringstyle=\ttfamily,showstringspaces=false,basicstyle=\footnotesize,captionpos=b}
\usepackage{xcolor}
\usepackage[newfloat]{minted}
\usepackage{caption}
\usepackage{fancyhdr}
\newenvironment{code}{\captionsetup{type=listing}}{}
\SetupFloatingEnvironment{listing}{name=Quellcode}
\setminted{
    linenos,
    frame=single,
    bgcolor=black!5,
    fontsize=\footnotesize
}
\usepackage{graphicx}  % Add this to your preamble

\usepackage[withpage]{acronym}

% Absatz ohne Einrückung
\setlength{\parskip}{1em} % 1em entspricht der Breite eines 'M'-Zeichens
\setlength{\parindent}{0pt}

% Indexerstellung
\makeindex

% Abkürzungsverzeichnis
\usepackage[german]{nomencl}
\let\abbrev\nomenclature

% Abkürzungsverzeichnis LiveTex Version
% Titel des Abkürzungsverzeichnisses
\renewcommand{\nomname}{Abkürzungsverzeichnis}
% Abstand zwischen Abkürzung und Erläuterung
\setlength{\nomlabelwidth}{.25\textwidth}
% Zwischenraum zwischen Abkürzung und Erläuterung mit Punkten
\renewcommand{\nomlabel}[1]{#1 \dotfill}
% Variation des Abstandes der einzelnen Abkürzungen zu einander
\setlength{\nomitemsep}{-\parsep}
% Index mit Abkürzungen erzeugen
\makenomenclature
%\makeglossary

% Abkürzungsverzeichnis TeTEX Version
% \usepackage[german]{nomencl}
% \makenomenclature
% %\makeglossary
% \renewcommand{\nomname}{Abkürzungsverzeichnis}
% \AtBeginDocument{\setlength{\nomlabelwidth}{.25\columnwidth}}
% \renewcommand{\nomlabel}[1]{#1 \dotfill}
% \setlength{\nomitemsep}{-\parsep}

% Optional: Einzelne Zeilen am Anfang einer Seite unterdrücken (Schusterjungen)
% \clubpenalty = 10000
% Optional: Einzelne Zeilen am Ende einer Seite unterdrücken (Hurenkinder)
% \widowpenalty = 10000
% \displaywidowpenalty = 10000

\begin{document}
% hier werden die Trennvorschläge inkludiert
%hier müssen alle Wörter rein, welche Latex von sich auch nicht korrekt trennt bzw. bei denen man die genaue Trennung vorgeben möchte
\hyphenation{
Film-pro-du-zen-ten
Lux-em-burg
Soft-ware-bau-steins
zeit-in-ten-siv
}


% Schriftart Helvetica verwenden
%\usepackage{helvet}
%\renewcommand\familydefault{\sfdefault}

% Leere Seite am Anfang
%\thispagestyle{empty} % erzeugt Seite ohne Kopf- / Fusszeile
%\mbox{}
%\newpage

% Titelseite %
% Cover page (Deckblatt)
% Uses variables defined in latex_einstellungen/variablen.tex
\thispagestyle{empty}

% Optional: Logo (place your file under assets/ and uncomment)
%\begin{figure}[t]
%  \centering
%  \includegraphics[width=0.25\textwidth]{assets/logo.pdf}
%\end{figure}

\vspace*{1.2cm}
\begin{center}
    {\normalsize \documentType}
\end{center}

\vspace{2.8cm}
\begin{center}
   
    {\Large \subjectDocument}\\[0.8em]
\end{center}

\vspace{5.5em}
\begin{center}
    \begingroup
    \small
    \ifdefempty{\university}{}{\university\par}
    \ifdefempty{\faculty}{}{\faculty\par}
    \ifdefempty{\program}{}{\program\par}
    
    \endgroup
\end{center}

\vspace{3em}
\begin{center}
    \textbf{Autor/in}\\
    \authorName\\
    geboren am \authorBirthdate\\
    Matrikelnummer: \studentId
\end{center}

\vspace{2em}
\begin{flushleft}
\begin{tabular}{@{}ll}
\textbf{Betreuung durch:} & \advisorA \\
                          & \advisorB \\
\end{tabular}
\end{flushleft}

\vfill

\begin{flushleft}
\begin{tabular}{@{}ll}
\textbf{Version vom:}    & \city, \today \\
\end{tabular}
\end{flushleft}


% römische Numerierung
\pagenumbering{Roman}

% 1.5 facher Zeilenabstand
\onehalfspacing

\newpage

% Einleitung / Abstract
\thispagestyle{empty}
\section*{Kurzfassung}

Diese Arbeit entwickelt und evaluiert ein FreeRTOS-basiertes System zur akustischen 
Entfernungsmessung zwischen zwei ESP32-S3-Knoten für den \emph{Außenbereich}. Ziel ist 
eine zuverlässige und kostengünstige Kurzstreckenmessung trotz Wind, Umgebungsgeräuschen 
und Mehrwegeffekten. Zwei identische Knoten mit I\textsuperscript{2}S-MEMS-Mikrofon und 
 Lautsprecher tauschen hörbare Chirps im Ping-Pong-Verfahren aus; ein Funklink synchronisiert 
 Startzeitpunkte und Zeitstempel. Zur Genauigkeitssteigerung kommen eine geeignete 
 Chirp-Charakteristik im oberen Hörbereich, eine temperaturgestützte 
 Schallgeschwindigkeitskorrektur, ein korrelationsbasiertes ToF-Verfahren mit 
 Sub-Sample-Peak-Schätzung sowie eine einmalige Verzögerungskalibrierung zum Einsatz. 
 Jitter wird durch ISR-Zeitstempel, I\textsuperscript{2}S-DMA und kerngebundene,
  priorisierte FreeRTOS-Tasks reduziert. Feldtests im Freien bestätigen die Eignung 
  hörfrequenzbasierter Audio-ToF-Messungen und bilden die Grundlage für robustere 
  Mehrknotensysteme.

\section*{Abstract}

This work develops and evaluates a FreeRTOS-based system for acoustic ranging between 
two ESP32-S3 nodes for \emph{outdoor} use. The goal is reliable, low-cost short-range 
measurements despite wind, ambient noise, and multipath. Two identical nodes equipped 
with an I\textsuperscript{2}S MEMS microphone and a loudspeaker exchange audible chirps 
in a ping-pong scheme; a radio link synchronizes start times and timestamps. To improve accuracy,
 the design employs a suitable chirp in the upper audible band, temperature-based speed-of-sound 
 correction (on-board sensor), cross-correlation with sub-sample peak estimation, and a one-time 
 delay calibration. Jitter is reduced through ISR-level timestamping, I\textsuperscript{2}S DMA,
  and core-pinned, prioritized FreeRTOS tasks. Outdoor field tests confirm the suitability of 
  audible-band audio ToF and provide a basis for more robust multi-node systems.


% einfacher Zeilenabstand
\singlespacing

\newpage
% Seitenzählung bei Inhaltsverzeichnis beginnen
\setcounter{page}{1}

% Inhaltsverzeichnis anzeigen
\thispagestyle{empty}
\vspace*{1.5cm}
\begin{center}
{\Large \textbf{Inhaltsverzeichnis}}
\end{center}

\vspace{2cm}
\noindent
\textbf{Kurzfassung} \dotfill II

\noindent
\textbf{Abstract} \dotfill III

\vspace{0.5cm}
\noindent
\textbf{1 Einleitung} \dotfill 1

\noindent
\hspace{1cm} 1.1 Motivation \dotfill 1

\noindent
\hspace{1cm} 1.2 Zielsetzung \dotfill 2

\noindent
\hspace{1cm} 1.3 Aufbau der Arbeit \dotfill 3

\vspace{0.5cm}
\noindent
\textbf{2 Grundlagen} \dotfill 4

\noindent
\hspace{1cm} 2.1 Begriffe und Definitionen \dotfill 4

\noindent
\hspace{1cm} 2.2 Stand der Technik \dotfill 6

\noindent
\hspace{1cm} 2.3 Relevante theoretische Konzepte \dotfill 8

\vspace{0.5cm}
\noindent
\textbf{3 Vorbereitung} \dotfill 10

\noindent
\hspace{1cm} 3.1 Anforderungsanalyse \dotfill 10

\noindent
\hspace{1cm} 3.2 Technische Rahmenbedingungen \dotfill 12

\noindent
\hspace{1cm} 3.3 Projektplanung \dotfill 14

\vspace{0.5cm}
\noindent
\textbf{4 Umsetzung} \dotfill 16

\noindent
\hspace{1cm} 4.1 Systemarchitektur \dotfill 16

\noindent
\hspace{1cm} 4.2 Implementierungsdetails \dotfill 18

\noindent
\hspace{1cm} 4.3 Herausforderungen und Lösungsansätze \dotfill 20

\vspace{0.5cm}
\noindent
\textbf{5 Evaluation} \dotfill 22

\noindent
\hspace{1cm} 5.1 Testumgebung und Methodik \dotfill 22

\noindent
\hspace{1cm} 5.2 Ergebnisse \dotfill 24

\noindent
\hspace{1cm} 5.3 Diskussion \dotfill 26

\vspace{0.5cm}
\noindent
\textbf{6 Zusammenfassung und Ausblick} \dotfill 28

\vspace{0.8cm}
\noindent
\textbf{Abbildungsverzeichnis} \dotfill II

\noindent
\textbf{Tabellenverzeichnis} \dotfill III

\noindent
\textbf{Quellcodeverzeichnis} \dotfill IV

\noindent
\textbf{Abkürzungsverzeichnis} \dotfill V

\noindent
\textbf{Literaturverzeichnis} \dotfill I

\noindent
\textbf{Anhang} \dotfill VI

\newpage
% das Abbildungsverzeichnis
% Verion 1: Abbildungsverzeichnis MIT führender Nummberierung endgueltig anzeigen
\listoffigures
% Abbildungsverzeichnis soll im Inhaltsverzeichnis auftauchen
\addcontentsline{toc}{section}{Abbildungsverzeichnis}

% Verion 2: Abbildungsverzeichnis OHNE führende Nummberierung endgueltig anzeigen
%\begingroup
%\renewcommand\numberline[1]{}
%\listoffigures
%\endgroup


% das Tabellenverzeichnis
\newpage
% \fancyhead[L]{Abbildungsverzeichnis / Abkürzungsverzeichnis} %Kopfzeile links
% Tabellenverzeichnis endgültig anzeigen
\listoftables
% Tabellenverzeichnis soll im Inhaltsverzeichnis auftauchen
\addcontentsline{toc}{section}{Tabellenverzeichnis}

% das Quellcodeverzeichnis
\newpage
\renewcommand*{\listlistingname}{Quellcodeverzeichnis}
\listoflistings % Add Quellcodeverzeichnis
\addcontentsline{toc}{section}{Quellcodeverzeichnis}

% das Abkürzungsverzeichnis
\newpage
% das Abkürzungsverzeichnis ausgeben
\fancyhead[L]{Abkürzungsverzeichnis} %Kopfzeile links
\section*{Abkürzungsverzeichnis}

\begin{acronym}
    \acro{API}{Application Programming Interface}
    \acro{CPU}{Central Processing Unit}
    % Fügen Sie weitere Abkürzungen hier hinzu
\end{acronym}



% \printnomenclature[3cm]
% Abkürzungsverzeichnis soll im Inhaltsverzeichnis auftauchen
\addcontentsline{toc}{section}{Abkürzungsverzeichnis}


%%%%%%% EINLEITUNG %%%%%%%%%%%%
\newpage
\fancyhead[L]{\nouppercase{\leftmark}} %Kopfzeile links

% 1,5 facher Zeilenabstand
\onehalfspacing

% arabische Seitennummerierung ab hier
\pagenumbering{arabic}

% Alternative Einbindung des Abstract in Kapitel "0" falls gewünscht
%\setcounter{section}{-1}
%\setcounter{page}{0}

% Option: Einbindung abstract
%\section*{Kurzfassung}

Diese Arbeit entwickelt und evaluiert ein FreeRTOS-basiertes System zur akustischen 
Entfernungsmessung zwischen zwei ESP32-S3-Knoten für den \emph{Außenbereich}. Ziel ist 
eine zuverlässige und kostengünstige Kurzstreckenmessung trotz Wind, Umgebungsgeräuschen 
und Mehrwegeffekten. Zwei identische Knoten mit I\textsuperscript{2}S-MEMS-Mikrofon und 
 Lautsprecher tauschen hörbare Chirps im Ping-Pong-Verfahren aus; ein Funklink synchronisiert 
 Startzeitpunkte und Zeitstempel. Zur Genauigkeitssteigerung kommen eine geeignete 
 Chirp-Charakteristik im oberen Hörbereich, eine temperaturgestützte 
 Schallgeschwindigkeitskorrektur, ein korrelationsbasiertes ToF-Verfahren mit 
 Sub-Sample-Peak-Schätzung sowie eine einmalige Verzögerungskalibrierung zum Einsatz. 
 Jitter wird durch ISR-Zeitstempel, I\textsuperscript{2}S-DMA und kerngebundene,
  priorisierte FreeRTOS-Tasks reduziert. Feldtests im Freien bestätigen die Eignung 
  hörfrequenzbasierter Audio-ToF-Messungen und bilden die Grundlage für robustere 
  Mehrknotensysteme.

\section*{Abstract}

This work develops and evaluates a FreeRTOS-based system for acoustic ranging between 
two ESP32-S3 nodes for \emph{outdoor} use. The goal is reliable, low-cost short-range 
measurements despite wind, ambient noise, and multipath. Two identical nodes equipped 
with an I\textsuperscript{2}S MEMS microphone and a loudspeaker exchange audible chirps 
in a ping-pong scheme; a radio link synchronizes start times and timestamps. To improve accuracy,
 the design employs a suitable chirp in the upper audible band, temperature-based speed-of-sound 
 correction (on-board sensor), cross-correlation with sub-sample peak estimation, and a one-time 
 delay calibration. Jitter is reduced through ISR-level timestamping, I\textsuperscript{2}S DMA,
  and core-pinned, prioritized FreeRTOS tasks. Outdoor field tests confirm the suitability of 
  audible-band audio ToF and provide a basis for more robust multi-node systems.

%\newpage

% einzelne Kapitel werden hier eingebunden
\section{Einleitung}

Führen Sie in das Thema ein, motivieren Sie die Arbeit und formulieren Sie die
Ziele. Skizzieren Sie den Aufbau des Dokuments. Dieser Text dient als
Platzhalter und sollte vollständig durch Ihre eigene Einleitung ersetzt werden.

\newpage

\section{Grundlagen}

Dieses Kapitel stellt die theoretischen und technischen Grundlagen vor, die zum Verständnis der akustischen Time-of-Flight-Messung im Außenbereich erforderlich sind.

\subsection{Schallausbreitung im Außenraum}

Die Ausbreitung von Schallwellen im Außenraum wird maßgeblich durch die physikalischen Eigenschaften der Luft sowie durch Umwelteinflüsse bestimmt. Die Schallgeschwindigkeit $c$ in trockener Luft lässt sich näherungsweise in Abhängigkeit von der Lufttemperatur $T$ in Grad Celsius durch die Beziehung
\[
c \approx 331{,}3 \, \text{m/s} + 0{,}6 \cdot T \, \text{m/s}
\]
beschreiben \cite{kuttruff2000raumakustik}. Temperaturänderungen wirken sich somit direkt auf die Laufzeitmessungen aus und müssen bei präzisen Time-of-Flight-Verfahren berücksichtigt werden. Neben der Temperatur beeinflussen auch Luftfeuchtigkeit und Luftdruck die Schallgeschwindigkeit, wenngleich in geringerem Maße \cite{bass1995atmospheric}.

Ein wesentlicher Aspekt der Schallausbreitung im Freien ist die Dämpfung mit zunehmender Entfernung. Diese setzt sich aus geometrischer Ausbreitung (Kugelausbreitung) und frequenzabhängiger atmosphärischer Absorption zusammen. Während die geometrische Dämpfung mit $1/r$ (bei Druckamplituden) bzw. $1/r^2$ (bei Intensitäten) beschrieben wird, nimmt die Absorption mit steigender Frequenz deutlich zu. Dies begrenzt die Reichweite insbesondere hochfrequenter akustischer Signale.

Darüber hinaus beeinflussen Umgebungsbedingungen die Ausbreitung erheblich. Wind kann durch Geschwindigkeitsgradienten zu einer Richtungsabhängigkeit der Schallgeschwindigkeit führen, wodurch Laufzeiten verzerrt werden. Turbulenzen verursachen zusätzlich Pegelschwankungen und Phasenmodulationen \cite{salomons2001computational}. Auch Boden- und Gebäudereflexionen führen zu Mehrwegeffekten, die bei Laufzeitmessungen als systematische Störgrößen in Erscheinung treten können. Diese Effekte sind im Kontext akustischer Entfernungsmessung besonders kritisch, da sie Fehlinterpretationen bei der Korrelation verursachen können.

Für die vorliegende Arbeit sind insbesondere zwei Punkte relevant: Erstens die temperaturabhängige Variation der Schallgeschwindigkeit, die bei Messungen im Außenraum durch eine entsprechende Korrektur kompensiert werden muss. Zweitens die Beeinflussung durch Mehrwegeffekte und atmosphärisches Rauschen, welche die Detektionswahrscheinlichkeit der gesendeten Signale reduzieren und die Erkennungsrate limitieren können.


\subsection{Signalgestaltung für akustische ToF-Verfahren}

\subsection{Korrelationsmethoden}


\subsection{Eigenschaften von ESP32-S3 und FreeRTOS}


\newpage

\section{Vorbereitung}

Beschreiben Sie hier die Vorarbeiten, Annahmen, Anforderungen und den
Projektkontext. Listen Sie relevante Werkzeuge, Datenquellen und Rahmenbedingungen
auf. Dieser Platzhaltertext sollte durch Ihre eigenen Inhalte ersetzt werden.

\newpage

\section{Umsetzung}

Beschreiben Sie die Implementierungsschritte, Architektur, Schnittstellen und
wesentliche Designentscheidungen. Referenzieren Sie Abbildungen, Tabellen und
Codeausschnitte nach Bedarf.



\newpage

\section{Evaluation}

Beschreiben Sie den Evaluationsaufbau, die Metriken und die Auswertung der
Ergebnisse. Fügen Sie ggf. Abbildungen und Tabellen hinzu.



\newpage

\section{Zusammenfassung und Ausblick}

Diese Arbeit hat die Entwicklung und Evaluierung eines FreeRTOS-basierten Systems zur akustischen Entfernungsmessung zwischen zwei ESP32-S3-Knoten für den Außenbereich erfolgreich durchgeführt.

\subsection{Zusammenfassung der Ergebnisse}



\subsection{Bewertung der Anforderungserfüllung}



\subsection{Ausblick auf zukünftige Arbeiten}


\newpage

\pagenumbering{Roman}
% einfacher Zeilenabstand
\singlespacing
% Literaturliste soll im Inhaltsverzeichnis auftauchen
\newpage
\phantomsection
\addcontentsline{toc}{section}{Literaturverzeichnis}
% Literaturverzeichnis anzeigen
\renewcommand\refname{Literaturverzeichnis}
\bibliography{Hauptdatei}

%% Index soll Stichwortverzeichnis heissen
% \newpage
% % Stichwortverzeichnis soll im Inhaltsverzeichnis auftauchen
% \addcontentsline{toc}{section}{Stichwortverzeichnis}
% \renewcommand{\indexname}{Stichwortverzeichnis}
% % Stichwortverzeichnis endgültig anzeigen
% \printindex

% \onehalfspacing
% % evtl. Anhang
% \newpage
% \phantomsection
% \addcontentsline{toc}{section}{Anhang}
% \fancyhead[L]{Anhang} %Kopfzeile links
% \input{anhang}

% \newpage
% \includepdf[pages=1, fitpaper=true]{flattened.pdf}

% \fancyhead[L]{Eigenständigkeitserklährung} %Kopfzeile links
% \input{eigenstaendigkeitserklaehrung} % deleted 

\end{document}
