% Festlegung des Allgemeinen Dokumentenformats
\documentclass[a4paper,12pt,headsepline]{scrartcl}

% Umlaute unter UTF8 nutzen
\usepackage[utf8]{inputenc}

% Variablen
% Variablen welche innerhalb der gesamten Arbeit zur Verfügung stehen sollen
% Titel und Thema
\newcommand{\titleDocument}{Projektarbeit}
\newcommand{\subjectDocument}{Entwicklung eines FreeRTOS-basierten Systems zur akustischen Entfernungsmessung im hörbaren Frequenzbereich}

% Dokumenttyp und Abschlussgrad
\newcommand{\documentType}{Projektarbeit}
\newcommand{\degree}{Bachelor of Science}
\newcommand{\degreeShort}{B. Sc.}

% Institution
\newcommand{\university}{An der Fachhochschule Dortmund}
\newcommand{\faculty}{im Fachbereich Informatik}
\newcommand{\program}{Studiengang Technische Informatik}

% Autorendaten
\newcommand{\authorName}{seifeddine Makhlouf}
\newcommand{\authorBirthdate}{20.01.2001}
\newcommand{\studentId}{999999}
\newcommand{\city}{Dortmund}

% Betreuung
\newcommand{\advisorA}{Prof. Dr. Frank Künemund}
\newcommand{\advisorB}{Dipl.-Ing. Dieter Zumkehr}


% weitere Pakete
% Grafiken aus PNG Dateien einbinden
\usepackage{graphicx}
\usepackage{tikz}

% Deutsche Sonderzeichen und Silbentrennung nutzen
\usepackage[ngerman]{babel}

% Eurozeichen einbinden
\usepackage[right]{eurosym}

% Zeichenencoding
\usepackage[T1]{fontenc}

\usepackage{lmodern}

% floatende Bilder ermöglichen
%\usepackage{floatflt}

\usepackage{float}

% mehrseitige Tabellen ermöglichen
\usepackage{longtable}

\usepackage{diagbox}
\usepackage{colortbl}

% Unterstützung für Schriftarten
%\newcommand{\changefont}[3]{ 
%\fontfamily{#1} \fontseries{#2} \fontshape{#3} \selectfont}

% Packet für Seitenrandabständex und Einstellung für Seitenränder
\usepackage{geometry}
\geometry{left=3.5cm, right=2cm, top=2.5cm, bottom=2cm}

% Paket für Boxen im Text
\usepackage{fancybox}

% bricht lange URLs "schön" um
\usepackage[hyphens,obeyspaces,spaces]{url}

% Paket für Textfarben
\usepackage{color}

% Mathematische Symbole importieren
\usepackage{amssymb}

\usepackage{amsmath}

% auf jeder Seite eine Überschrift (alt, zentriert)
%\pagestyle{headings}

\usepackage{pdfpages}
\usepackage{etoolbox}

% erzeugt Inhaltsverzeichnis mit Querverweisen zu den Abschnitten (PDF Version)
\usepackage[bookmarksnumbered,pdftitle={\titleDocument},hyperfootnotes=false]{hyperref}
%\hypersetup{colorlinks, citecolor=red, linkcolor=blue, urlcolor=black}
%\hypersetup{colorlinks, citecolor=black, linkcolor= black, urlcolor=black}

% neue Kopfzeilen mit fancypaket
\usepackage{fancyhdr} %Paket laden
\pagestyle{fancy} %eigener Seitenstil
\fancyhf{} %alle Kopf- und Fußzeilenfelder bereinigen
\fancyhead[L]{\nouppercase{\leftmark}} %Kopfzeile links
\fancyhead[C]{} %zentrierte Kopfzeile
\fancyhead[R]{\thepage} %Kopfzeile rechts
\renewcommand{\headrulewidth}{0.4pt} %obere Trennlinie
%\fancyfoot[C]{\thepage} %Seitennummer
%\renewcommand{\footrulewidth}{0.4pt} %untere Trennlinie

% change font to serif for headings - buggy
% \usepackage{sectsty}
% \allsectionsfont{\normalfont\bfseries}

% für Tabellen
\usepackage{array}

% Runde Klammern für Zitate
%\usepackage[numbers,round]{natbib}

% Festlegung Art der Zitierung - Havardmethode: Abkuerzung Autor + Jahr
\bibliographystyle{alphadin}

% Schaltet den zusätzlichen Zwischenraum ab, den LaTeX normalerweise nach einem Satzzeichen einfügt.
%\frenchspacing

% Paket für Zeilenabstand
\usepackage{setspace}

% für Bildbezeichner
\usepackage{capt-of}

% für Stichwortverzeichnis
\usepackage{makeidx}

%Konfiguriere das Inhaltsverzeichnis
\usepackage{tocbasic}
\DeclareTOCStyleEntries[
  raggedentrytext,
  numwidth=0pt,
  numsep=1ex,
  dynnumwidth,
]{tocline}{chapter,section,subsection,subsubsection,paragraph,subparagraph}
\DeclareTOCStyleEntries[
  indent=0pt,
  linefill=\TOCLineLeaderFill,
]{tocline}{section,subsection,subsubsection,paragraph,subparagraph}


% für Listings
% \usepackage{listings}
% \lstset{numbers=left, numberstyle=\tiny, numbersep=5pt, keywordstyle=\color{black}\bfseries, stringstyle=\ttfamily,showstringspaces=false,basicstyle=\footnotesize,captionpos=b}
\usepackage{xcolor}
\usepackage[newfloat]{minted}
\usepackage{caption}
\usepackage{fancyhdr}
\newenvironment{code}{\captionsetup{type=listing}}{}
\SetupFloatingEnvironment{listing}{name=Quellcode}
\setminted{
    linenos,
    frame=single,
    bgcolor=black!5,
    fontsize=\footnotesize
}
\usepackage{graphicx}  % Add this to your preamble

\usepackage[withpage]{acronym}

% Absatz ohne Einrückung
\setlength{\parskip}{1em} % 1em entspricht der Breite eines 'M'-Zeichens
\setlength{\parindent}{0pt}

% Indexerstellung
\makeindex

% Abkürzungsverzeichnis
\usepackage[german]{nomencl}
\let\abbrev\nomenclature

% Abkürzungsverzeichnis LiveTex Version
% Titel des Abkürzungsverzeichnisses
\renewcommand{\nomname}{Abkürzungsverzeichnis}
% Abstand zwischen Abkürzung und Erläuterung
\setlength{\nomlabelwidth}{.25\textwidth}
% Zwischenraum zwischen Abkürzung und Erläuterung mit Punkten
\renewcommand{\nomlabel}[1]{#1 \dotfill}
% Variation des Abstandes der einzelnen Abkürzungen zu einander
\setlength{\nomitemsep}{-\parsep}
% Index mit Abkürzungen erzeugen
\makenomenclature
%\makeglossary

% Abkürzungsverzeichnis TeTEX Version
% \usepackage[german]{nomencl}
% \makenomenclature
% %\makeglossary
% \renewcommand{\nomname}{Abkürzungsverzeichnis}
% \AtBeginDocument{\setlength{\nomlabelwidth}{.25\columnwidth}}
% \renewcommand{\nomlabel}[1]{#1 \dotfill}
% \setlength{\nomitemsep}{-\parsep}

% Optional: Einzelne Zeilen am Anfang einer Seite unterdrücken (Schusterjungen)
% \clubpenalty = 10000
% Optional: Einzelne Zeilen am Ende einer Seite unterdrücken (Hurenkinder)
% \widowpenalty = 10000
% \displaywidowpenalty = 10000

\begin{document}
% hier werden die Trennvorschläge inkludiert
%hier müssen alle Wörter rein, welche Latex von sich auch nicht korrekt trennt bzw. bei denen man die genaue Trennung vorgeben möchte
\hyphenation{
Film-pro-du-zen-ten
Lux-em-burg
Soft-ware-bau-steins
zeit-in-ten-siv
}


% Schriftart Helvetica verwenden
%\usepackage{helvet}
%\renewcommand\familydefault{\sfdefault}

% Leere Seite am Anfang
%\thispagestyle{empty} % erzeugt Seite ohne Kopf- / Fusszeile
%\mbox{}
%\newpage

% Titelseite %
% Cover page (Deckblatt)
% Uses variables defined in latex_einstellungen/variablen.tex
\thispagestyle{empty}

% Optional: Logo (place your file under assets/ and uncomment)
%\begin{figure}[t]
%  \centering
%  \includegraphics[width=0.25\textwidth]{assets/logo.pdf}
%\end{figure}

\vspace*{1.2cm}
\begin{center}
    {\normalsize \documentType}
\end{center}

\vspace{2.8cm}
\begin{center}
   
    {\Large \subjectDocument}\\[0.8em]
\end{center}

\vspace{5.5em}
\begin{center}
    \begingroup
    \small
    \ifdefempty{\university}{}{\university\par}
    \ifdefempty{\faculty}{}{\faculty\par}
    \ifdefempty{\program}{}{\program\par}
    
    \endgroup
\end{center}

\vspace{3em}
\begin{center}
    \textbf{Autor/in}\\
    \authorName\\
    geboren am \authorBirthdate\\
    Matrikelnummer: \studentId
\end{center}

\vspace{2em}
\begin{flushleft}
\begin{tabular}{@{}ll}
\textbf{Betreuung durch:} & \advisorA \\
                          & \advisorB \\
\end{tabular}
\end{flushleft}

\vfill

\begin{flushleft}
\begin{tabular}{@{}ll}
\textbf{Version vom:}    & \city, \today \\
\end{tabular}
\end{flushleft}


% römische Numerierung
\pagenumbering{Roman}

% 1.5 facher Zeilenabstand
\onehalfspacing

\newpage

% Einleitung / Abstract
\thispagestyle{empty}
\section*{Kurzfassung}

Diese Arbeit entwickelt und evaluiert ein FreeRTOS-basiertes System zur akustischen 
Entfernungsmessung zwischen zwei ESP32-S3-Knoten für den \emph{Außenbereich}. Ziel ist 
eine zuverlässige und kostengünstige Kurzstreckenmessung trotz Wind, Umgebungsgeräuschen 
und Mehrwegeffekten. Zwei identische Knoten mit I\textsuperscript{2}S-MEMS-Mikrofon und 
 Lautsprecher tauschen hörbare Chirps im Ping-Pong-Verfahren aus; ein Funklink synchronisiert 
 Startzeitpunkte und Zeitstempel. Zur Genauigkeitssteigerung kommen eine geeignete 
 Chirp-Charakteristik im oberen Hörbereich, eine temperaturgestützte 
 Schallgeschwindigkeitskorrektur, ein korrelationsbasiertes ToF-Verfahren mit 
 Sub-Sample-Peak-Schätzung sowie eine einmalige Verzögerungskalibrierung zum Einsatz. 
 Jitter wird durch ISR-Zeitstempel, I\textsuperscript{2}S-DMA und kerngebundene,
  priorisierte FreeRTOS-Tasks reduziert. Feldtests im Freien bestätigen die Eignung 
  hörfrequenzbasierter Audio-ToF-Messungen und bilden die Grundlage für robustere 
  Mehrknotensysteme.

\section*{Abstract}

This work develops and evaluates a FreeRTOS-based system for acoustic ranging between 
two ESP32-S3 nodes for \emph{outdoor} use. The goal is reliable, low-cost short-range 
measurements despite wind, ambient noise, and multipath. Two identical nodes equipped 
with an I\textsuperscript{2}S MEMS microphone and a loudspeaker exchange audible chirps 
in a ping-pong scheme; a radio link synchronizes start times and timestamps. To improve accuracy,
 the design employs a suitable chirp in the upper audible band, temperature-based speed-of-sound 
 correction (on-board sensor), cross-correlation with sub-sample peak estimation, and a one-time 
 delay calibration. Jitter is reduced through ISR-level timestamping, I\textsuperscript{2}S DMA,
  and core-pinned, prioritized FreeRTOS tasks. Outdoor field tests confirm the suitability of 
  audible-band audio ToF and provide a basis for more robust multi-node systems.


% einfacher Zeilenabstand
\singlespacing

\newpage
% Seitenzählung bei Inhaltsverzeichnis beginnen
\setcounter{page}{1}

% Inhaltsverzeichnis anzeigen
\thispagestyle{empty}
\vspace*{1.5cm}
\begin{center}
{\Large \textbf{Erweitertes Inhaltsverzeichnis}}
\end{center}

\vspace{2cm}
\noindent
\textbf{Kurzfassung} \dotfill II

\noindent
\textbf{Abstract} \dotfill III

\vspace{0.5cm}
\noindent
\textbf{Einleitung} \dotfill 1

\noindent
\hspace{1cm} 3.1 Motivation \dotfill 1

\noindent
\hspace{1cm} 3.2 Zielsetzung \dotfill 2

\noindent
\hspace{1cm} 3.3 Aufbau der Arbeit \dotfill 3

\vspace{0.5cm}
\noindent
\textbf{Grundlagen} \dotfill 4

\noindent
\hspace{1cm} 4.1 Schallausbreitung im Außenraum \dotfill 4

\noindent
\hspace{1cm} 4.2 Signalgestaltung für akustische ToF-Verfahren \dotfill 6

\noindent
\hspace{1cm} 4.3 Korrelationsmethoden \dotfill 8

\noindent
\hspace{1cm} 4.4 Eigenschaften von ESP32‑S3 und FreeRTOS \dotfill 10

\vspace{0.5cm}
\noindent
\textbf{Anforderungen} \dotfill 12

\noindent
\hspace{1cm} 5.1 Anwendungsszenarien \dotfill 12

\noindent
\hspace{1cm} 5.2 Funktionale Anforderungen  \dotfill 14

\noindent
\hspace{1cm} 5.3 Nicht-funktionale Anforderungen  \dotfill 16

\vspace{0.5cm}
\noindent
\textbf{Hardware} \dotfill 18

\noindent
\hspace{1cm} 6.1 Prototypenaufbau (Steckbrett) \dotfill 18

\noindent
\hspace{1cm} 6.2 Lautsprecher- und Mikrofonwahl \dotfill 20

\noindent
\hspace{1cm} 6.3 Funkmodulauswahl \dotfill 22


\noindent
\hspace{1cm} 6.6 Schaltplan (Schematik) \dotfill 28

\noindent
\hspace{1cm} 6.7 PCB-Layout \dotfill 30

\vspace{0.5cm}
\noindent
\textbf{Software} \dotfill 28

\noindent
\hspace{1cm} 7.1 FreeRTOS-Architektur  \dotfill 28

\noindent
\hspace{1cm} 7.2 I²S-DMA \dotfill 29

\noindent
\hspace{1cm} 7.3 Funkbasierte Synchronisation \dotfill 30

\noindent
\hspace{1cm} 7.4 Maßnahmen zur Jitterreduktion \dotfill 31

\vspace{0.5cm}
\noindent
\textbf{Signaldesign} \dotfill 32

\noindent
\hspace{1cm} 8.1 Chirp-Charakteristik, Fensterung und Bandbegrenzung \dotfill 32

\noindent
\hspace{1cm} 8.2 Vorverarbeitung \dotfill 33

\noindent
\hspace{1cm} 8.3 Korrelationsbasierte Laufzeitschätzung mit Sub-Sample-Peak-Refinement \dotfill 34

\noindent
\hspace{1cm} 8.4 Temperaturkompensation \dotfill 35

\noindent
\hspace{1cm} 8.5 Verzögerungskalibrierung \dotfill 36

\vspace{0.5cm}
\noindent
\textbf{Methodik} \dotfill 37



\vspace{0.5cm}
\noindent
\textbf{Ergebnisse} \dotfill 52

%

\vspace{0.5cm}
\noindent
\textbf{Zusammenfassung und Ausblick} \dotfill 64

\vspace{0.8cm}
\noindent
\textbf{Abbildungsverzeichnis} \dotfill II

\noindent
\textbf{Tabellenverzeichnis} \dotfill III

\noindent
\textbf{Quellcodeverzeichnis} \dotfill IV

\noindent
\textbf{Abkürzungsverzeichnis} \dotfill V

\noindent
\textbf{Literaturverzeichnis} \dotfill I

\noindent
\textbf{Anhang} \dotfill VI

\newpage
% das Abbildungsverzeichnis
% Verion 1: Abbildungsverzeichnis MIT führender Nummberierung endgueltig anzeigen
\listoffigures
% Abbildungsverzeichnis soll im Inhaltsverzeichnis auftauchen
\addcontentsline{toc}{section}{Abbildungsverzeichnis}

% Verion 2: Abbildungsverzeichnis OHNE führende Nummberierung endgueltig anzeigen
%\begingroup
%\renewcommand\numberline[1]{}
%\listoffigures
%\endgroup


% das Tabellenverzeichnis
\newpage
% \fancyhead[L]{Abbildungsverzeichnis / Abkürzungsverzeichnis} %Kopfzeile links
% Tabellenverzeichnis endgültig anzeigen
\listoftables
% Tabellenverzeichnis soll im Inhaltsverzeichnis auftauchen
\addcontentsline{toc}{section}{Tabellenverzeichnis}

% das Quellcodeverzeichnis
\newpage
\renewcommand*{\listlistingname}{Quellcodeverzeichnis}
\listoflistings % Add Quellcodeverzeichnis
\addcontentsline{toc}{section}{Quellcodeverzeichnis}

% das Abkürzungsverzeichnis
\newpage
% das Abkürzungsverzeichnis ausgeben
\fancyhead[L]{Abkürzungsverzeichnis} %Kopfzeile links
\section*{Abkürzungsverzeichnis}

\begin{acronym}
    \acro{API}{Application Programming Interface}
    \acro{CPU}{Central Processing Unit}
    % Fügen Sie weitere Abkürzungen hier hinzu
\end{acronym}



% \printnomenclature[3cm]
% Abkürzungsverzeichnis soll im Inhaltsverzeichnis auftauchen
\addcontentsline{toc}{section}{Abkürzungsverzeichnis}


%%%%%%% EINLEITUNG %%%%%%%%%%%%
\newpage
\fancyhead[L]{\nouppercase{\leftmark}} %Kopfzeile links

% 1,5 facher Zeilenabstand
\onehalfspacing

% arabische Seitennummerierung ab hier
\pagenumbering{arabic}

% Alternative Einbindung des Abstract in Kapitel "0" falls gewünscht
%\setcounter{section}{-1}
%\setcounter{page}{0}

% Option: Einbindung abstract
%\section*{Kurzfassung}

Diese Arbeit entwickelt und evaluiert ein FreeRTOS-basiertes System zur akustischen 
Entfernungsmessung zwischen zwei ESP32-S3-Knoten für den \emph{Außenbereich}. Ziel ist 
eine zuverlässige und kostengünstige Kurzstreckenmessung trotz Wind, Umgebungsgeräuschen 
und Mehrwegeffekten. Zwei identische Knoten mit I\textsuperscript{2}S-MEMS-Mikrofon und 
 Lautsprecher tauschen hörbare Chirps im Ping-Pong-Verfahren aus; ein Funklink synchronisiert 
 Startzeitpunkte und Zeitstempel. Zur Genauigkeitssteigerung kommen eine geeignete 
 Chirp-Charakteristik im oberen Hörbereich, eine temperaturgestützte 
 Schallgeschwindigkeitskorrektur, ein korrelationsbasiertes ToF-Verfahren mit 
 Sub-Sample-Peak-Schätzung sowie eine einmalige Verzögerungskalibrierung zum Einsatz. 
 Jitter wird durch ISR-Zeitstempel, I\textsuperscript{2}S-DMA und kerngebundene,
  priorisierte FreeRTOS-Tasks reduziert. Feldtests im Freien bestätigen die Eignung 
  hörfrequenzbasierter Audio-ToF-Messungen und bilden die Grundlage für robustere 
  Mehrknotensysteme.

\section*{Abstract}

This work develops and evaluates a FreeRTOS-based system for acoustic ranging between 
two ESP32-S3 nodes for \emph{outdoor} use. The goal is reliable, low-cost short-range 
measurements despite wind, ambient noise, and multipath. Two identical nodes equipped 
with an I\textsuperscript{2}S MEMS microphone and a loudspeaker exchange audible chirps 
in a ping-pong scheme; a radio link synchronizes start times and timestamps. To improve accuracy,
 the design employs a suitable chirp in the upper audible band, temperature-based speed-of-sound 
 correction (on-board sensor), cross-correlation with sub-sample peak estimation, and a one-time 
 delay calibration. Jitter is reduced through ISR-level timestamping, I\textsuperscript{2}S DMA,
  and core-pinned, prioritized FreeRTOS tasks. Outdoor field tests confirm the suitability of 
  audible-band audio ToF and provide a basis for more robust multi-node systems.

%\newpage

% einzelne Kapitel werden hier eingebunden
\section{Einleitung}

Führen Sie in das Thema ein, motivieren Sie die Arbeit und formulieren Sie die
Ziele. Skizzieren Sie den Aufbau des Dokuments. Dieser Text dient als
Platzhalter und sollte vollständig durch Ihre eigene Einleitung ersetzt werden.

\newpage

\section{Grundlagen}

Dieses Kapitel stellt die theoretischen und technischen Grundlagen vor, die zum Verständnis der akustischen Time-of-Flight-Messung im Außenbereich erforderlich sind.

\subsection{Schallausbreitung im Außenraum}

Die Ausbreitung von Schallwellen im Außenraum wird maßgeblich durch die physikalischen Eigenschaften der Luft sowie durch Umwelteinflüsse bestimmt. Die Schallgeschwindigkeit $c$ in trockener Luft lässt sich näherungsweise in Abhängigkeit von der Lufttemperatur $T$ in Grad Celsius durch die Beziehung
\[
c \approx 331{,}3 \, \text{m/s} + 0{,}6 \cdot T \, \text{m/s}
\]
beschreiben \cite{kuttruff2000raumakustik}. Temperaturänderungen wirken sich somit direkt auf die Laufzeitmessungen aus und müssen bei präzisen Time-of-Flight-Verfahren berücksichtigt werden. Neben der Temperatur beeinflussen auch Luftfeuchtigkeit und Luftdruck die Schallgeschwindigkeit, wenngleich in geringerem Maße \cite{bass1995atmospheric}.

Ein wesentlicher Aspekt der Schallausbreitung im Freien ist die Dämpfung mit zunehmender Entfernung. Diese setzt sich aus geometrischer Ausbreitung (Kugelausbreitung) und frequenzabhängiger atmosphärischer Absorption zusammen. Während die geometrische Dämpfung mit $1/r$ (bei Druckamplituden) bzw. $1/r^2$ (bei Intensitäten) beschrieben wird, nimmt die Absorption mit steigender Frequenz deutlich zu. Dies begrenzt die Reichweite insbesondere hochfrequenter akustischer Signale.

Darüber hinaus beeinflussen Umgebungsbedingungen die Ausbreitung erheblich. Wind kann durch Geschwindigkeitsgradienten zu einer Richtungsabhängigkeit der Schallgeschwindigkeit führen, wodurch Laufzeiten verzerrt werden. Turbulenzen verursachen zusätzlich Pegelschwankungen und Phasenmodulationen \cite{salomons2001computational}. Auch Boden- und Gebäudereflexionen führen zu Mehrwegeffekten, die bei Laufzeitmessungen als systematische Störgrößen in Erscheinung treten können. Diese Effekte sind im Kontext akustischer Entfernungsmessung besonders kritisch, da sie Fehlinterpretationen bei der Korrelation verursachen können.

Für die vorliegende Arbeit sind insbesondere zwei Punkte relevant: Erstens die temperaturabhängige Variation der Schallgeschwindigkeit, die bei Messungen im Außenraum durch eine entsprechende Korrektur kompensiert werden muss. Zweitens die Beeinflussung durch Mehrwegeffekte und atmosphärisches Rauschen, welche die Detektionswahrscheinlichkeit der gesendeten Signale reduzieren und die Erkennungsrate limitieren können.


\subsection{Signalgestaltung für akustische ToF-Verfahren}

\subsection{Korrelationsmethoden}


\subsection{Eigenschaften von ESP32-S3 und FreeRTOS}


\newpage

\section{Anforderungen}

Dieses Kapitel leitet aus den Anwendungsszenarien die funktionalen und nicht-funktionalen Anforderungen für das akustische \ac{ToF}-System ab.

\subsection{Anwendungsszenarien}

Das primäre Anwendungsszenario des entwickelten Systems ist die akustische Distanzmessung zwischen zwei im Außenraum positionierten Knoten. Jeder Knoten besteht aus einem ESP32-S3, einem Lautsprecher zur Signalerzeugung sowie einem Mikrofon zur Signalaufnahme. Die Kommunikation und Synchronisation zwischen den Knoten erfolgt drahtlos über Funk . 

\textbf{Funktionsweise:}  
Die Messung basiert auf einem zweistufigen Ablauf. Zunächst sendet der Sender-Knoten ein kurzes Funksignal an den Empfänger-Knoten. Dieses Signal dient der \textit{Synchronisation} und informiert den Empfänger über den unmittelbar bevorstehenden akustischen Sendevorgang. Direkt nach dem Aussenden des Funksignals erzeugt der Sender das akustische Chirp-Signal und gibt es über den Lautsprecher aus. Der Empfänger-Knoten startet exakt mit dem Eintreffen des Funksignals seine Aufnahme über das Mikrofon und zeichnet damit das ankommende Chirp-Signal auf. 

Die Zeitdifferenz zwischen dem Sendezeitpunkt (definiert durch das Funksignal) und dem Empfangszeitpunkt des akustischen Signals wird anschließend bestimmt. Dies geschieht durch Kreuzkorrelation des empfangenen Signals mit der bekannten Referenz des Chirps. Aus der Lage des Korrelationsmaximums ergibt sich die Laufzeitdifferenz, die unter Berücksichtigung der temperaturabhängigen Schallgeschwindigkeit in eine Distanz umgerechnet wird.  

Durch diese Kombination aus Funksynchronisation und akustischer Signalauswertung wird sichergestellt, dass die Messung nicht durch Taktabweichungen oder Jitter der beiden Knoten verfälscht wird. Das Funksignal reduziert die Unsicherheit beim Startzeitpunkt der Aufnahme, während das akustische Signal die eigentliche Distanzinformation liefert.  

\begin{figure}[h]
    \centering
    \includegraphics[width=0.9\linewidth]{Szenario mit zwei Knoten im Außenraum.png}
    \caption{Anwendungsszenario: Distanzmessung zwischen zwei Knoten mit Funksynchronisation und akustischem \ac{ToF}.}
    \label{fig:scenario}
\end{figure}


Das beschriebene Anwendungsszenario bildet die Grundlage für die nachfolgenden funktionalen und nicht-funktionalen Anforderungen.



\subsection{Funktionale Anforderungen}

\begin{verbatim}
-------------------------------------
1. Signalerzeugung (Chirp)
-------------------------------------
Description         : Der Sender muss akustische Chirp-Signale im Bereich 
                      von 2–8 \ac{kHz} mit einer Dauer von 20–60 ms 
                      zuverlässig erzeugen können.
Verification Method : Funktionstest der Signalerzeugung durch Messung mit
                      Oszilloskop und Spektralanalyse.
Category            : Functional

-------------------------------------
2. Signalaufnahme
-------------------------------------
Description         : Der Empfänger muss akustische Signale im gleichen Frequenzbereich 
                      mit Abtastraten bis 48 \ac{kHz} aufnehmen können.
Verification Method : Test der Mikrofon-Schnittstelle und Validierung der Abtastrate 
                      über Software- und Hardwaremessung.
Category            : Functional

-------------------------------------
3. Funksynchronisation
-------------------------------------
Description         : Das System muss eine drahtlose Synchronisation der Knoten 
                      (z. B. ESP-NOW) unterstützen, sodass der Empfänger zeitgenau 
                      mit der Aufnahme beginnt.
Verification Method : Messung der Zeitdifferenz zwischen Funksignal und Start 
                      der Aufnahme, Analyse mit Logic Analyzer.
Category            : Functional

-------------------------------------
4. Signalverarbeitung (Korrelation)
-------------------------------------
Description         : Der Empfänger muss die aufgenommenen Signale in Echtzeit 
                      mittels Kreuzkorrelation mit einer Referenz analysieren, 
                      um den Time-of-Flight zu bestimmen.
Verification Method : Softwaretest der Korrelation auf bekannten Referenzdaten, 
                      Validierung der Laufzeitbestimmung.
Category            : Functional

-------------------------------------
5. Distanzberechnung
-------------------------------------
Description         : Das System muss die Laufzeitdifferenz in eine Entfernung d 
                      umrechnen, unter Berücksichtigung der temperaturabhängigen 
                      Schallgeschwindigkeit.
Verification Method : Vergleich mit bekannten Referenzdistanzen unter verschiedenen 
                      Temperaturbedingungen.
Category            : Functional

-------------------------------------
6. Datenkommunikation
-------------------------------------
Description         : Die berechnete Distanz muss lokal gespeichert oder über Funk 
                      an andere Knoten oder eine zentrale Einheit übermittelt werden.
Verification Method : Funktionstest der Datenspeicherung und Funkübertragung 
                      zwischen den Knoten.
Category            : Functional

-------------------------------------
7. Messbereich
-------------------------------------
Description         : Das System muss Distanzen bis mindestens 25 m zuverlässig messen können.
Verification Method : Feldtests mit bekannten Referenzabständen im Bereich 1–25 m.
Category            : Functional
\end{verbatim}


\subsection{Nicht-funktionale Anforderungen}


\begin{verbatim}
-------------------------------------
1. Signalerzeugung (Chirp)
-------------------------------------
Description         : Der Sender muss akustische Chirp-Signale im Bereich 
                      von 2–8 \ac{kHz} mit einer Dauer von 20–60 ms 
                      zuverlässig erzeugen können.
Verification Method : Funktionstest der Signalerzeugung durch Messung mit 
                      Oszilloskop und Spektralanalyse.
Category            : Functional

-------------------------------------
2. Signalaufnahme
-------------------------------------
Description         : Der Empfänger muss akustische Signale im gleichen 
                      Frequenzbereich mit Abtastraten bis 48 \ac{kHz} 
                      aufnehmen können.
Verification Method : Test der Mikrofon-Schnittstelle und Validierung der 
                      Abtastrate über Software- und Hardwaremessung.
Category            : Functional

-------------------------------------
3. Funksynchronisation
-------------------------------------
Description         : Das System muss eine drahtlose Synchronisation der 
                      Knoten (z. B. ESP-NOW) unterstützen, sodass der 
                      Empfänger zeitgenau mit der Aufnahme beginnt.
Verification Method : Messung der Zeitdifferenz zwischen Funksignal und 
                      Start der Aufnahme, Analyse mit Logic Analyzer.
Category            : Functional

-------------------------------------
4. Signalverarbeitung (Korrelation)
-------------------------------------
Description         : Der Empfänger muss die aufgenommenen Signale in 
                      Echtzeit mittels Kreuzkorrelation mit einer Referenz 
                      analysieren, um den Time-of-Flight zu bestimmen.
Verification Method : Softwaretest der Korrelation auf bekannten 
                      Referenzdaten, Validierung der Laufzeitbestimmung.
Category            : Functional

-------------------------------------
5. Distanzberechnung
-------------------------------------
Description         : Das System muss die Laufzeitdifferenz in eine 
                      Entfernung d umrechnen, unter Berücksichtigung der 
                      temperaturabhängigen Schallgeschwindigkeit.
Verification Method : Vergleich mit bekannten Referenzdistanzen unter 
                      verschiedenen Temperaturbedingungen.
Category            : Functional

-------------------------------------
6. Datenkommunikation
-------------------------------------
Description         : Die berechnete Distanz muss lokal gespeichert oder 
                      über Funk an andere Knoten oder eine zentrale Einheit 
                      übermittelt werden.
Verification Method : Funktionstest der Datenspeicherung und 
                      Funkübertragung zwischen den Knoten.
Category            : Functional

-------------------------------------
7. Messbereich
-------------------------------------
Description         : Das System muss Distanzen bis mindestens 25 m 
                      zuverlässig messen können.
Verification Method : Feldtests mit bekannten Referenzabständen im 
                      Bereich 1–25 m.
Category            : Functional
\end{verbatim}


\subsection{Nicht-funktionale Anforderungen}

\begin{verbatim}
-------------------------------------
1. Robustheit gegenüber Störungen
-------------------------------------
Description         : Das System muss Umgebungsgeräusche (Wind, Verkehr, 
                      Tiere) und Mehrwegeffekte tolerieren und dennoch 
                      eine zuverlässige Peak-Erkennung ermöglichen.
Verification Method : Feldtests unter realen Outdoor-Bedingungen mit 
                      künstlich hinzugefügtem Störgeräusch.
Category            : Non-Functional

-------------------------------------
2. Messgenauigkeit
-------------------------------------
Description         : Die Distanzmessung soll eine Genauigkeit im Bereich 
                      von ± 0,1 m oder besser erreichen.
Verification Method : Vergleich mit Referenzmessungen mittels 
                      Laser-Distanzsensor.
Category            : Non-Functional

-------------------------------------
3. Energieeffizienz
-------------------------------------
Description         : Das System muss für batteriebetriebenen Einsatz 
                      optimiert sein, indem Rechen- und Funkoperationen 
                      energieeffizient geplant werden.
Verification Method : Messung des Stromverbrauchs während verschiedener 
                      Betriebsmodi.
Category            : Non-Functional

-------------------------------------
4. Echtzeitfähigkeit
-------------------------------------
Description         : Die gesamte Messung (Signalerzeugung, Aufnahme, 
                      Korrelation, Berechnung) soll innerhalb von weniger 
                      als 100 ms abgeschlossen sein.
Verification Method : Messung der End-to-End-Latenz mit Oszilloskop oder 
                      Logic Analyzer.
Category            : Non-Functional

-------------------------------------
5. Portabilität und Skalierbarkeit
-------------------------------------
Description         : Die Knoten sollen kompakt, leicht und in größeren 
                      Netzwerken einsetzbar sein.
Verification Method : Überprüfung des Formfaktors sowie Netzwerktests mit 
                      mehr als zwei Knoten.
Category            : Non-Functional

-------------------------------------
6. Wartungsarmut
-------------------------------------
Description         : Das System soll im Außenraum über längere Zeiträume 
                      ohne manuelle Justierung oder Kalibrierung zuverlässig 
                      arbeiten.
Verification Method : Langzeittest im Freien unter wechselnden 
                      Umweltbedingungen.
Category            : Non-Functional
\end{verbatim}



\newpage

\section{Hardware}

Dieses Kapitel beschreibt den Hardwareentwurf vom Steckbrettprototyp bis zur PCB-Implementierung, einschließlich der Komponentenauswahl und des Schaltungsdesigns.

\subsection{Prototypenaufbau (Steckbrett)}


\subsection{Lautsprecher- und Mikrofonwahl}

\subsection{Funkmodulauswahl}

\subsection{Verstärker- und Versorgungsdesign}


\subsection{Schaltplan (Schematik)}

\subsection{PCB-Layout}

\newpage

\section{Software}

Dieses Kapitel erläutert die Softwarearchitektur unter FreeRTOS, einschließlich Task-Design, I²S-Verarbeitung und Synchronisationsstrategien.

\subsection{FreeRTOS-Architektur}


\subsection{I²S-DMA}



\subsection{Maßnahmen zur Jitterreduktion}


\newpage

\section{Signaldesign}

Dieses Kapitel behandelt das Design der akustischen Signale, die Signalverarbeitung und die Laufzeitschätzung für die ToF-Messung.

\subsection{Chirp-Charakteristik, Fensterung und Bandbegrenzung}



\subsection{Korrelationsbasierte Laufzeitschätzung mit Sub-Sample-Peak-Refinement}



\subsection{Temperaturkompensation}


\subsection{Verzögerungskalibrierung}


\newpage

\section{Methodik}

Dieses Kapitel definiert die Messmethodik für die experimentelle Bewertung des akustischen ToF-Systems, einschließlich Testaufbau, Versuchspläne und Auswerteverfahren.

\subsection{Testaufbau und -umgebung (Außenbereich)}

\subsection{Metriken und Auswerteverfahren}

\newpage

\section{Ergebnisse}

Dieses Kapitel präsentiert die Messergebnisse der experimentellen Evaluierung des akustischen ToF-Systems und diskutiert diese im Kontext der definierten Anforderungen.

\subsection{Reichweite}


\subsection{Erkennungsrate}


\subsection{Fehlerkennwerte}



\newpage

\section{Zusammenfassung und Ausblick}

Diese Arbeit hat die Entwicklung und Evaluierung eines FreeRTOS-basierten Systems zur akustischen Entfernungsmessung zwischen zwei ESP32-S3-Knoten für den Außenbereich erfolgreich durchgeführt.

\subsection{Zusammenfassung der Ergebnisse}



\subsection{Bewertung der Anforderungserfüllung}



\subsection{Ausblick auf zukünftige Arbeiten}


\newpage

\pagenumbering{Roman}
% einfacher Zeilenabstand
\singlespacing
% Literaturliste soll im Inhaltsverzeichnis auftauchen
\newpage
\phantomsection
\addcontentsline{toc}{section}{Literaturverzeichnis}
% Literaturverzeichnis anzeigen
\renewcommand\refname{Literaturverzeichnis}
\bibliography{Hauptdatei}

%% Index soll Stichwortverzeichnis heissen
% \newpage
% % Stichwortverzeichnis soll im Inhaltsverzeichnis auftauchen
% \addcontentsline{toc}{section}{Stichwortverzeichnis}
% \renewcommand{\indexname}{Stichwortverzeichnis}
% % Stichwortverzeichnis endgültig anzeigen
% \printindex

% \onehalfspacing
% % evtl. Anhang
% \newpage
% \phantomsection
% \addcontentsline{toc}{section}{Anhang}
% \fancyhead[L]{Anhang} %Kopfzeile links
% \input{anhang}

% \newpage
% \includepdf[pages=1, fitpaper=true]{flattened.pdf}

% \fancyhead[L]{Eigenständigkeitserklährung} %Kopfzeile links
% \input{eigenstaendigkeitserklaehrung} % deleted 

\end{document}
